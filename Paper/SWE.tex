\documentclass{elsarticle}
\usepackage{elf}

\author{Erich L Foster}

\begin{document}
  \begin{frontmatter}
    \author[1]{Erich L Foster\corref{cor1}}
    \ead{efoster@bcamath.org}
    \ead[url]{http://www.bcamath.org/en/people/efoster}
    \cortext[cor1]{corresponding author}

    \author[1]{Johan Jansson}
    \ead{iliescu@vt.edu}
    \ead[url]{http://www.bcamath.org/en/people/jjansson}

    \address[1]{Basque Center for Applied Mathematics, Alameda Mazarredo, 14,
      48009 Bilbao, Basque Country -- Spain}

    \title{An adaptive DNS/LES for a Shallow Water Equation model of the Ocean}

    \begin{abstract}
      Meso-scale eddies are present in only small portions of the oceans,
      however they contain significant portions of the oceans' kinetic energy.
      For this reason it is important to resolve meso-scale eddies, however
      meso-scale eddies are of the order of 10km, and thus for a structured grid
      general ocean circulation model, requiring a rigid mesh size, it is not
      feasible to completely resolve such features on any of todays computers.
      While the Finite Element Method can be used on unstructured grids, and
      therefore allows for the easy refinement of meshes, for resolving such
      features as meso-scale eddies and western boundary currents, which move
      about continuously, a dynamic approach to mesh refinement is necessary. In
      this paper we develop an Adaptive DNS/LES model for the Shallow Water
      Equations of the Ocean, which allows for resolving important features
      while under-resolving features seen as less important.
    \end{abstract}
  \end{frontmatter}

  \section{Introduction} \label{sec:Intro}
  The oceans have two important roles in the climate system:
\begin{inparaenum}[1)]\item They store and release heat seasonally, and \item
they transport heat around the globe by their large scale currents
\end{inparaenum} These currents play a significant roll in climate dynamics
by, in the case of sub-tropic gyres, transporting the warm waters of the
tropics poleward and in the case of sub-polar gyres transporting the cold
waters of the pole southward where they mix with the warm waters from the
tropics.

The first attempts at modelling circulation of the ocean were made after Ekman
(1905) developed his theory of Ekman pumping \cite{Ekman1905}. Later Sverdrup
(1947) developed a simple theory that relates the wind stress curl to ocean
currents \cite{Sverdrup1947}, followed by the development of simple analytical
models of wind induced ocean currents by Stommel (1948), Munk (1950), and
Charney (1948) \cite{Stommel1948,Munk1950,Charney1948}. Then in 1963 the first
numerical model of ocean circulation was developed by Bryan et al.
\cite{Bryan1963}. Since the work developed by Bryan there have been many
numerical studies of ocean circulation, however the majority of which have
relied on the Finite Difference Method (FDM). For a more detailed review see
Griffies et al. \cite{Griffies2000}. However, some \emph{Finite Element} ocean
models do exist, see for example, \cite{Fix} and \cite{Myers}. In addition to
the research stated above one can find: the QUODDY model by Ip et. al.
\cite{Ip1995}, a 3D shallow water equations model based on linear triangular
elements; the DGCOM model by Giraldo et al. \cite{GiraldoWWW}, a 2D shallow
water equation model based on the discontinuous Galerkin approach.

The Finite Element Method (FEM) offers some distinct advantages over the FDM:
the FEM allows for the easy representation of coastlines; and the FDM suffers
from an inherent rigidness of structured grids making refinement in specific
regions difficult. On the other hand the FEM allows for the use of
unstructured grids, which allows for grid refinement in areas for which
dynamics of the system have been deemed important, such as narrow straits,
islands, areas containing western boundary currents, or meso-scale eddies.
Meso-scale eddies are present in only small portions of the oceans, however
they contain significant portions of the oceans' kinetic energy. For this
reason it is important to resolve meso-scale eddies, however to resolve these
meso-scale eddies a mesh size on the order of 10km is needed. For a structured
grid, requiring a rigid mesh size, this is not feasible on any of todays
computers. Thus, being able to refine the mesh in particular areas allows for
meso-scale eddies to be resolved. Other methods which allow for unstructured
grids include techniques such as the finite volumes, finite elements, and
spectral methods. 

On the other hand, while the FEM can be used on unstructured grids, and
therefore allows for the easy refinement of meshes, for resolving such
features as meso-scale eddies and western boundary currents which move about
continuously a dynamic approach to mesh refinement is necessary.  However, it
is not known a priori where such eddies may occur and therefore an \emph{a
posteriori adaptive mesh refinement} scheme such as the method developed by
Bab{\v{s}}ka et al. in \cite{Babuska1978} can be used to refine a mesh in the
areas for which it is necessary. In particular we intend to adapt the method
developed by Hoffman et al. in \cite{Hoffman2004} to ocean models. 

The scheme developed in \cite{Hoffman2004} is referred to, by the authors, as
an \emph{Adaptive DNS/LES} where certain features of the flow are resolved in a
\emph{Direct Numerical Simulation} (DNS), while other features are left
unresolved and modelled by \emph{Large Eddy Simulation} (LES). In the case of
the LES model, the residual based stabilization is used as the sub-grid model.
This method has been shown to be very efficient at high Reynolds numbers
($Re>10^6$) \cite{Jansson2011}. This allows one to only apply an LES model
where the contribution of error in a desired quantity is small and therefore
not significant to the desired quantity.

However, the effects of rotation are expected to introduce new challenges and
therefore it is unlikely one can simply apply this method without some
adaptations. The majority of problems this method has been applied have been
for aerodynamics and thus many of the dynamic features of concern are not
continuously moving. Therefore, the authors haven't been concerned with mesh
coarsening. The added feature of mesh coarsening will be essential for the
efficient simulations of the world's oceans.


  \bibliographystyle{plain}
  \bibliography{Complete}

\end{document}
