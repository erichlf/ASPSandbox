In this section we introduce the usage of AFES by the use of examples. The AFES
api is written solely in Python (with the exception of UFL for the weak form)
and thus the user is expected to have a basic knowledge of Python programming.
However, since AFES is based on the FEniCS/DOLFIN Python interface it does not
compromise on performance\cite{Alnae2011}. The use of DOLFIN's Python interface
allows for greater ease of use, as compared to the C++ interface. Eventually, we
intend to offer a GUI which would require the use of no programming and only
entering the details for the domain, boundary conditions, and the weak form.

\subsection{Solving a PDE} \label{sse:Solver}

    In this subsection we explain how to solve a basic time dependent PDE using
    the AFES api. We choose to explain the API through example; we choose to
    make things quite simple at first and so the first example is the Heat
    equation.  Since, the FEM requires the weak form of any PDE we assume the
    user understands how to determine the weak for a given PDE and thus no
    explanation of the weak form is given. For a very basic explanation of the
    weak form see \cite[Chapter 6.2.2]{Eriksson2009}.

    Of course to be able to solve any PDE certain assumptions must be made. AFES
    makes the assumption that an $H^1$ FE is appropriate for solving the given
    PDE, i.e. the FE discretization is stable. Additionally, AFES assumes the

    \begin{equation}
        R(u) = u_t - F(u;t) = 0.
        \label{eq:BasicForm}
    \end{equation}
    AFES also assumes that the given PDE can be discretized in time in the
    following way
    \begin{equation}
        r(u^n, u^n, v) = \frac{1}{k} (u^n - u^{n-1}, v) - (F(\hat{u};t^n), v) = 0,
        \label{eq:WeakResidual}
    \end{equation}
    where $u^n,\, u^{n-1},\, \hat{u}$, and $v$ are the solution at the
    $n^{th}$-timestep, the solution at the $(n-1)^{st}$-timestep,
    $\theta$-weighted average of the solution for the $n^{th}$ and
    $(n-1)^{th}$-timestep, i.e.
    \begin{equation}
        \hat{u} = \theta\, u^n + (1 - \theta)\, u^{n-1},
        \label{eq:uAvg}
    \end{equation}
    and the test function, respectively.

\subsubsection{Heat Equation} \label{sss:Heat}
    The heat equation is a canonical example of parabolic partial differential
    equations. Given the domain $\Omega$ with boundary $\partial \Omega$ and
    time interval $I$ the heat equation is given by
    \begin{equation}
        \begin{split}
            u_t - \nabla \left( \kappa \nabla u \right) &= f
                \qquad (\mathbf{x}, t)\in \Omega \times I \\
            u(\mathbf{x}, 0) &= u_0(\mathbf{x}) \quad \mathbf{x} \in \Omega \\
            u(\mathbf{x}, t) &= g(\mathbf{x},t) \quad (\mathbf{x}, t)\in
                \partial\Omega \times I, \\
        \end{split}
        \label{eq:Heat}
    \end{equation}
    where $f$ is the internal heat source/sink and $g$ is the boundary
    condition. Taking our FE basis functions $v_h\in V^h\subset H^1_0(\Omega)$
    then we are trying to find a $u^n_h \in V^h \subset H^1(\Omega)$ such that
    \begin{equation}
        \frac{1}{k} (u^n - u^{n-1}, v_h) + (\kappa \nabla \hat{u}, \nabla v_h) =
        0 \quad \forall v_h \in V^h,
        \label{eq:WeakHeat}
    \end{equation}
    where $k$ is the time step, $u^n_h$ is the FE solution at the
    $n^{th}$-timestep, $u^{n-1}_h$ is the FE solution at the
    $(n-1)^{st}$-timestep, and $\kappa$ is the heat coefficient. To further
    specify the problem we will define the domain to be the unit square, i.e.
    \begin{equation}
        \Omega := [0, 1] \times [0, 1],
        \label{eq:HeatDomain}
    \end{equation}
    define the function to be
    \begin{equation}
        f(x,y,t) = \sin(pi\, x
        \label{<+label+>}
    \end{equation}<++>

\subsection{Goal Oriented Adaptivity} \label{sse:Adaptivity}

\subsubsection{Navier-Stokes Equation} \label{sss:NSE}

\subsection{Optimization} \label{sse:Optimization}
