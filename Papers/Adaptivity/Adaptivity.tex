\documentclass[a4paper]{article}
\PassOptionsToPackage{firstinits=true,isbn=false,url=false,doi=false,style=numeric,sorting=ydnt}{biblatex}
\usepackage{elf}

%keywords environment
\def\keywords{\vspace{.5em}
{\textit{Keywords}:\,\relax%
}}
\def\endkeywords{\par}

\addbibresource{Complete.bib}

\author[1]{Erich L Foster}

\author[1,2]{Johan Jansson}

\affil[1]{Basque Center for Applied Mathematics, Bilbao, Basque Country -- Spain}

\affil[2]{CSC, KTH Royal Institute of Technology, Stockholm, Sweden}

\title{Automated Error Control in Finite Elements for Time-Dependent Problems}

\begin{document}
  \maketitle
  \tableofcontents
  \begin{abstract}
    In this paper we present a new approach to \emph{automated goal-oriented
    error control} to the solutions of \emph{linear and non-linear
    time-dependent} partial differential equations. The approach is based on
    using the \emph{a posteriori error estimate} directly, where both the primal
    and dual solutions are approximated using the same finite element space.
    Traditionally, the a posteriori error estimate has been thought to contain
    no information about the error, since it gives a global a posteriori error
    estimate that is zero due to \emph{Galerkin Orthogonality}. However, we show
    that locally, on each triangle, the orthogonal error representation behaves
    similar to the non-orthogonal error representation and thus the orthogonal
    error representation is rich in information.
  \end{abstract}
  \begin{keywords}
    Finite elements, a posteriori, error control, adaptivity, fluid dynamics
  \end{keywords}

  \section{Introduction} \label{sec:Intro}
  Partial differential equations (PDEs) describe many models in accross all fields
of science, while the Finite element method (FEM) is one of the most powerful
methods for solving PDEs. However, the FEM can be both difficult to program and
difficult for many to fully grasp. Additionally, many of the associated tasks
with solving time dependent (and steady state) PDEs are the similar (if not the
same) for most PDEs and can be automated. Furthermore, many scientists may not
be familiar with programming. Thus, a package which can simplify the process of
implementing finite element (FE) solutions, including the reduction of
repetitive tasks and the amount of programming required, would be very useful.

The FEniCS project\cite{Alnae2011} set out to address many of these issues, by
offering various programming libraries (in both C++ and Python). However, the
FEniCS project, as of yet, does not automate the time-stepping procedure for
PDEs. Additionally, many of the required steps for implementing a FE solver in
FEniCS are quite repetitive and could be automated. The \ASP library sets out to
further automate the solutions of PDEs and hopefully will simplify the process.
Currently, the user is still required to do some basic programming, but we
intend to reduce this to either none or a very minimal amount.

The \ASP software library offers a simple Python programming interface, built
upon FEniCS, which will greatly reduce the amount of programming required by the
user and completely eliminate the necessity for creating a simple time stepper.
However, the user is required to setup the problem domain, boundary conditions,
forcing functions, and weak residual using the FEniCS/DOLFIN Python interface.

In addition to automating the solution to time dependent PDEs, \ASP offers a
novel automated goal oriented mesh adaptivity \cite{Foster2014e, Jansson2014a,
Jansson2014b}.  The goal oriented mesh adaptivity code is based on solving a
dual problem.  This dual problem is constructed and solved using DOLFIN-Adjoint
\cite{Ham2012, Ferrell2014}, an automated adjointing framework for FEniCS, and
thus needs very little user input.  Currently, the only additional task the user
must perform, for implementing the adaptivity, is defining a goal functional.
In this way the weak form of the PDE does not change at all and the user need
only to add the definition of a goal functional.


  \section{Algorithm} \label{sec:Algorithm}
  
\begin{algorithm}[!htp]
  \caption{Adaptive mesh refinement} \label{alg:Adaptivity}
  \KwData{Let $(\mathcal{T}_k, V_k,\mathcal{P})$ be a finite element
    discretization of $r(U,v) = a(U,v) - L(v)$.} \vspace{0.5em}
  \SetAlgoLined
  \While{ $\left|\sum_{j=1}^M \mathcal{E}^{cG(1)}_{K_j}\right| > TOL$}{
      \nl For the mesh $\mathcal{T}_k$: compute the primal problem and the dual problem. \\
      \nl Mark some chosen percentage of the elements with highest
        $\mathcal{E}^{cG(1)}_{K_j}$ for refinement.\\
      \nl Generate the refined mesh $\mathcal{T}^{k+1}$, set $k = k + 1$
  }
\end{algorithm}


  \section{Results} \label{sec:Results}
  In this section we present a detailed computational study of the automated
goal-oriented adaptive algorithm presented in \autoref{sec:Algorithm} applied to
some time-dependent test problems. To this end we first we apply our algorithm
to a linear problem, \autoref{tst:Heat}, where the error representation is
orthogonal due to Galerkin orthogonality. Namely, we demonstrate the
effectiveness of \autoref{alg:Adaptivity} on the heat equation. Next we
demonstrate the effectiveness of \autoref{alg:Adaptivity} on non-linear
problems, where the error representation is no longer orthogonal. To this end,
we apply the algorithm to the benchmark problems presented in Sch\"afer et
al.\cite{Schaefer1996}, \autoref{tst:Cylinder2D}, \autoref{tst:Rectangle2D}, and
\autoref{tst:Rectangle3D}, where the equation of interest is the Icompressible
Navier-Stokes equation and our particular focus is on flow around bluff bodies.
Finally, in \autoref{tst:rhoNSE}, we apply our algorithm to the ever more
complex problem of Variable Density Incompressible Navier-Stokes equation, where
we focus on the solution to Rayleigh-Taylor instability.

For each test below we set the max iteration to $20$ and set the $TOL=0$. In
this way we will treat the solution obtained from the final adaptive mesh as the
so-called 'true solution' and use this solution to determine the effectivity of
the adaptive algorithm.

\subsection{Orthogonal Error Representation}
In this subsection we demonstrate the effectiveness of \autoref{alg:Adaptivity}
for problems where the global error representation is orthogonal. In these types
of problems the traditional thought is that the error representation doesn't
provide information about the error. However, in \autoref{tst:Heat}, we show
that the local error representation does, indeed, contain significant
information about the error and thus is an effective way of controlling the
error.

\begin{test}[Anisotropic Heat Equation] \label{tst:Heat}
  The heat equation is a purely parabolic linear partial differential equation
  and therefore the global error representation is zero due to Galerkin
  Orthogonality.  Thus it is an excellent choice to demonstrate the
  effectiveness of the algorithm described in \autoref{sec:Algorithm}.

  In this test problem we extend the test given by Li et al.
  \cite[Example 5.1]{Li2010}, i.e. the anisotropic diffusion equation, to the
  time dependent case. Thus we are concerned with the numerical solution of the
  anisotropic heat equation on a domain with a hole given by
  \begin{equation} \label{eq:Heat}
    \begin{split}
      u_t - \nabla \cdot (\kappa \nabla u) &= f, \quad \text{in } \Omega \\
      u &= g, \quad \text{on } \partial \Omega \\
      u(0;x,y) &= u_0(x,y)
    \end{split}
  \end{equation}
  where $\Omega = \Omega_1\setminus~\Omega_2$ is the physical domain, $f$ is the
  forcing function, $g$ is the function describing the boundary condition, and
  $\kappa$ is the diffusion tensor. For this test we take $\Omega_1 = [0,
  1]^2$, $\Omega_2 = [\frac{4}{9}, \frac{5}{9}]^2$, $f = 0$
  \begin{equation}
    g = \begin{cases}
        0 & \text{if } (x,y) \in \partial \Omega_1 \\
        0.5 - 0.5\, \cos(10\, t) & \text{if } (x,y) \in \partial \Omega_2,
    \end{cases}
    \label{eqn:BCFunction}
  \end{equation}
  and the diffusion tensor to be
  \begin{equation}
    \kappa = \begin{bmatrix} 500.5 & 499.5 \\ 499.5 & 500.5 \end{bmatrix}.
    \label{eqn:DiffusionTensor}
  \end{equation}
  The above diffusion tensor results in a diffusion process which is
  preferential along the line $x = y$.
  \textcolor{red}{Need info on discretization.}
\end{test}

\subsection{Non-Orthogonal Error Representation}
In this subsection we demonstrate the effectiveness of \autoref{alg:Adaptivity}
applied to problems where the global error representation is no longer
orthogonal. In particular, we look at applying \autoref{alg:Adaptivity} to the
Incompressible Navier-Stokes and the Variable Density Incompressible Navier
Stokes. In both cases, we solve each test case using a
Galerkin\slash~Least-Squares (GLS) stabilized cG(1)cG(1) finite element, where
the first cG(1) indicates that the test and trial functions are both piecewise
linear, while the second cG(1) indicates that the test piecewise constants and
the trial functions are continuous piecewise linear.

\subsubsection{Incompressible Navier-Stokes}
For test problems \ref{tst:Cylinder2D}, \ref{tst:Rectangle2D}, and
\ref{tst:Rectangle3D} we are concerned with the Incompressible Navier-Stokes
with constant kinematic viscosity, $\nu>0$, in a domain $\Omega\subset \R^d$,
with boundary $\partial \Omega$, i.e.
where the strong residual is given by
\begin{equation}
    \begin{split}
      \mathbf{u}_t + \left( \mathbf{u} \cdot \nabla \right) \mathbf{u} - \nu\,
          \Delta \mathbf{u} + \nabla p = \mathbf{f}, \quad \mathbf{x} \in \Omega \\
          \nabla \cdot \mathbf{u} = 0, \quad \mathbf{x} \in \Omega
    \end{split}
  \label{eqn:NSE}
\end{equation}
Given a step size $k$ and applying the cG(1)cG(1) finite element discretization
to \eqref{eqn:NSE} with $V = (v, q) \in X \subset [H^1_0(\Omega)]^d \times
H^1(\Omega)$ gives the following weak residual
\begin{equation}
  \begin{split}
    r(\bar{U}^n; V) &= \left(\mathbf{u}^n - \mathbf{u}^{n-1}\right)\,k^{-1}
        + (\left( \bar{\mathbf{u}}^n \cdot \nabla \right) \bar{\mathbf{u}}^n, v) \\
        &\quad+ \nu\, (\nabla \bar{\mathbf{u}}^n, \nabla v)
        - (p^n, \nabla \cdot v) - (\mathbf{f}, v)
        + (\nabla \cdot \bar{\mathbf{u}}^n, v) = 0
  \end{split}
  \label{eqn:WeakNSE}
\end{equation}
where $\bar{U}^n = (\bar{\mathbf{u}}^n,p)$, and $\bar{\mathbf{u}}^n =
\frac{1}{2}\left(\mathbf{u}^n + \mathbf{u}^{n-1}\right)$. Given the solution
$U=(\mathbf{u},p)$, since the elements we are concerned with (cG(1)cG(1)) have
test functions which are linear in space and constant in time the strong
residual is given by
\begin{equation}
    R(\bar{U}^n,U) = \begin{cases}
      \left(\mathbf{u} \cdot \nabla \right) \bar{\mathbf{u}}^n
        + \nabla p^n - \mathbf{f} = 0 \\
      \nabla \cdot \bar{\mathbf{u}}^n = 0.
    \end{cases}
  \label{eqn:StrongNSE}
\end{equation}
\textcolor{red}{Should $\mathbf{f}$ be an average between time steps?}
Finally, the GLS-cG(1)cG(1) discretization of \eqref{eqn:NSE} is given by
\begin{equation}
  r(\bar{U}^n,V) + SD_{\delta}^n(\bar{U}^n,V) = 0, \quad \forall V=(v,q) \in X.
  \label{eqn:G2}
\end{equation}
Here $SD_{\delta}^n$ is the GLS and is given by
\begin{equation}
  SD_{\delta}^n \equiv
    \delta_1 (\left(\bar{\mathbf{u}}^n \cdot \nabla \right) \bar{\mathbf{u}}^n
        + \nabla p^n - \mathbf{f},
      \left(\bar{\mathbf{u}}^n \cdot \nabla \right) v + \nabla q)
      + \delta_2 (\nabla \cdot \bar{\mathbf{U}}^n, \nabla \cdot v),
  \label{eqn:NSEStabilization}
\end{equation}
where $\delta_1 = \kappa_1 (k^{-1} + |\mathbf{u}^{n-1}|^2\, h^{-2})^{-1/2}$, and
$\delta_2 = \kappa_2 h$, while $\kappa_1$ and $\kappa_2$ are problem independent
constants of unit size.


\begin{test}[Flow Around a Cylinder] \label{tst:Cylinder2D}
\end{test}

\begin{test}[2D Flow Around a Square] \label{tst:Rectangle2D}
\end{test}

\begin{test}[3D Flow Around a Rectangular-Cylinder] \label{tst:Rectangle3D}
\end{test}

\subsubsection{Variable Density Incompressible Navier-Stokes}
\begin{test}[Variable Density Navier-Stokes] \label{tst:rhoNSE}
\end{test}


  \section{Conclusions} \label{sec:Conclusions}
  In this report we presented a new approach to automate goal-oriented error
control to solutions of partial differential equations. We demonstrated the
effectiveness of this approach on linear and non-linear problems. In particular,
we show that the method is more effective when the chosen functional is a
measure of local phenomenon as opposed to a more global phenomenon, since the
adaptivity turns out to be similar to uniform refinement.



  \section*{Acknowledgements}
  This research has been supported by EU-FET grant EUNISON 308874, the European
  Research Council, the Swedish Foundation for Strategic Research, the Swedish
  Research Council, the Basque Excellence Research Center (BERC) program by
  the Basque Government, and the Severo Ochoa excellence award by the Spanish
  Government.

  \printbibliography
\end{document}
