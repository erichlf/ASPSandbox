\documentclass[a4paper]{article}
\PassOptionsToPackage{firstinits=true,isbn=false,url=false,doi=false,style=numeric,sorting=nyt}{biblatex}
\usepackage{elf}

%keywords environment
\def\keywords{\vspace{.5em}
{\textit{Keywords}:\,\relax%
}}
\def\endkeywords{\par}

\addbibresource{Complete.bib}

\author[1]{Erich L Foster}

\author[1,2]{Johan Jansson}

\affil[1]{Basque Center for Applied Mathematics, Bilbao, Basque Country -- Spain}

\affil[2]{CSC, KTH Royal Institute of Technology, Stockholm, Sweden}

\title{Automated Error Control in Finite Elements for Time-Dependent Problems}

\begin{document}
  \maketitle
  \tableofcontents
  \begin{abstract}
    In this paper we present a new approach to \emph{automated goal-oriented
    error control} to the solutions of \emph{linear and non-linear
    time-dependent} partial differential equations. The approach is based on
    using the \emph{a posteriori error estimate} directly, where both the primal
    and dual solutions are approximated using the same finite element space.
    Traditionally, the a posteriori error estimate has been thought to contain
    no information about the error, since it gives a global a posteriori error
    estimate that is zero due to \emph{Galerkin Orthogonality}. However, we show
    that locally, on each triangle, the orthogonal error representation behaves
    similar to the non-orthogonal error representation and thus the orthogonal
    error representation is rich in information.
  \end{abstract}
  \begin{keywords}
    Finite elements, a posteriori, error control, adaptivity, fluid dynamics
  \end{keywords}

  \section{Introduction} \label{sec:Intro}
  The oceans have two important roles in the climate system:
\begin{inparaenum}[1)]\item They store and release heat seasonally, and \item
they transport heat around the globe by their large scale currents
\end{inparaenum} These currents play a significant roll in climate dynamics
by, in the case of sub-tropic gyres, transporting the warm waters of the
tropics poleward and in the case of sub-polar gyres transporting the cold
waters of the pole southward where they mix with the warm waters from the
tropics.

The first attempts at modelling circulation of the ocean were made after Ekman
(1905) developed his theory of Ekman pumping \cite{Ekman1905}. Later Sverdrup
(1947) developed a simple theory that relates the wind stress curl to ocean
currents \cite{Sverdrup1947}, followed by the development of simple analytical
models of wind induced ocean currents by Stommel (1948), Munk (1950), and
Charney (1948) \cite{Stommel1948,Munk1950,Charney1948}. Then in 1963 the first
numerical model of ocean circulation was developed by Bryan et al.
\cite{Bryan1963}. Since the work developed by Bryan there have been many
numerical studies of ocean circulation, however the majority of which have
relied on the Finite Difference Method (FDM). For a more detailed review see
Griffies et al. \cite{Griffies2000}. However, some \emph{Finite Element} ocean
models do exist, see for example, \cite{Fix} and \cite{Myers}. In addition to
the research stated above one can find: the QUODDY model by Ip et. al.
\cite{Ip1995}, a 3D shallow water equations model based on linear triangular
elements; the DGCOM model by Giraldo et al. \cite{GiraldoWWW}, a 2D shallow
water equation model based on the discontinuous Galerkin approach.

The Finite Element Method (FEM) offers some distinct advantages over the FDM:
the FEM allows for the easy representation of coastlines; and the FDM suffers
from an inherent rigidness of structured grids making refinement in specific
regions difficult. On the other hand the FEM allows for the use of
unstructured grids, which allows for grid refinement in areas for which
dynamics of the system have been deemed important, such as narrow straits,
islands, areas containing western boundary currents, or meso-scale eddies.
Meso-scale eddies are present in only small portions of the oceans, however
they contain significant portions of the oceans' kinetic energy. For this
reason it is important to resolve meso-scale eddies, however to resolve these
meso-scale eddies a mesh size on the order of 10km is needed. For a structured
grid, requiring a rigid mesh size, this is not feasible on any of todays
computers. Thus, being able to refine the mesh in particular areas allows for
meso-scale eddies to be resolved. Other methods which allow for unstructured
grids include techniques such as the finite volumes, finite elements, and
spectral methods. 

On the other hand, while the FEM can be used on unstructured grids, and
therefore allows for the easy refinement of meshes, for resolving such
features as meso-scale eddies and western boundary currents which move about
continuously a dynamic approach to mesh refinement is necessary.  However, it
is not known a priori where such eddies may occur and therefore an \emph{a
posteriori adaptive mesh refinement} scheme such as the method developed by
Bab{\v{s}}ka et al. in \cite{Babuska1978} can be used to refine a mesh in the
areas for which it is necessary. In particular we intend to adapt the method
developed by Hoffman et al. in \cite{Hoffman2004} to ocean models. 

The scheme developed in \cite{Hoffman2004} is referred to, by the authors, as
an \emph{Adaptive DNS/LES} where certain features of the flow are resolved in a
\emph{Direct Numerical Simulation} (DNS), while other features are left
unresolved and modelled by \emph{Large Eddy Simulation} (LES). In the case of
the LES model, the residual based stabilization is used as the sub-grid model.
This method has been shown to be very efficient at high Reynolds numbers
($Re>10^6$) \cite{Jansson2011}. This allows one to only apply an LES model
where the contribution of error in a desired quantity is small and therefore
not significant to the desired quantity.

However, the effects of rotation are expected to introduce new challenges and
therefore it is unlikely one can simply apply this method without some
adaptations. The majority of problems this method has been applied have been
for aerodynamics and thus many of the dynamic features of concern are not
continuously moving. Therefore, the authors haven't been concerned with mesh
coarsening. The added feature of mesh coarsening will be essential for the
efficient simulations of the world's oceans.


  \section{Algorithm} \label{sec:Algorithm}
  In this section we initially assume a general linear time-dependent variational
problem of the form:
\begin{equation}
    \begin{split}
        &\text{find } u \in V \text{ such that} \\
        r(u, v) &:= (u_t, v) + a(u, v) - L(v) = 0 \quad \forall v \in V,
    \end{split}
    \label{eq:WeakResidual}
\end{equation}
where $a(\cdot, \cdot)$ and $L(v)$ are the bilinear and linear forms associated
with the variational problem, while $r(\cdot, \cdot)$ is the weak-residual.

Additionally, given the primal problem, \eqref{eq:WeakResidual}, then we assume
the dual solution is given by:
\begin{equation*}
    \begin{split}
        &\text{find } \Phi \in V^* \text{ such that} \\
        r(v, \Phi) &:= (u_t, \Phi) + a(v, \Phi) - (v, \psi) = 0
                       \quad \forall v \in V,
    \end{split}
\end{equation*}
where $\psi$ is the projection of the goal functional $M(\cdot) = (\cdot, \psi)$
onto the finite element space $V$. In other words, given the solution $u$ to
\eqref{eq:WeakResidual} we solve the problem
\begin{equation}
    \begin{split}
        &\text{find } \Phi \in V^* \text{ such that} \\
        r(v, \Phi) &:= (u_t, \Phi) + a(v, \Phi) - M(u) = 0
                       \quad \forall v \in V.
    \end{split}
    \label{eq:Dual}
\end{equation}

Given the solutions $u,\, u^h$ to the primal and discrete primal problems,
respectively, while $\Phi$ is the solution to the dual problem then
taking $v$ and $v^h$ to be $u$ and $u^h$, respectively, we can determine the
error in the goal functional in the following manner
\begin{align}
    M(u) - M(u^h) &= \left((u_t, \Phi) + a(u, \Phi)\right)
                      - \left( (u^h_t, \Phi) + a(u^h, \Phi) \right) \nonumber \\
    M(u - u^h) &= (u_t - u^h_t, \Phi) + a(u - u^h, \Phi) \nonumber \\
    M(e) &= 0 \label{eq:DualError}
\end{align}
where the last line, \eqref{eq:DualError} is due to Galerkin orthogonality. On
the other hand we have
\begin{align}
    M(e) & = \left((u_t, \Phi) + a(u, \Phi) - L(\Phi)\right)
             - \left((u^h_t, \Phi) + a(u^h, \Phi) - L(\Phi) \right) \nonumber \\
    &= -\left((u^h_t, \Phi) + a(u^h, \Phi) - L(\Phi) \right) \nonumber \\
    &= -r(u^h, \Phi) \label{eq:ErrorRepresentation} \\
    |M(e)| &= |r(u^h, \Phi)| \label{eq:ErrorEstimate}
\end{align}
Thus, we have an error estimate, \eqref{eq:ErrorEstimate}, and it appears that
the determining the error in a goal functional, $M(e)$, is equivalent to
evaluating the weak residual, \eqref{eq:WeakResidual}, using the primal and dual
solutions. However, due to Galerkin orthogonality the error representation,
\eqref{eq:ErrorRepresentation} evaluates to zero if the dual solution, $\Phi$,
and the primal solution, $u^h$, are in the same space.

\begin{algorithm}[!htp]
\begin{minipage}[t]{\columnwidth} %this minipage is a hack to use footenotes
  \caption{Adaptive mesh refinement} \label{alg:Adaptivity}
  \KwData{Let $(\mathcal{T}_h, V^h,\mathcal{P})$ be a finite element
    discretization of $r(u^h ,v^h) = (u^h_t, v^h) + a(u^h, v^h) - L(v^h)$.}
    \vspace{0.5em}
  \SetAlgoLined
  \While{ $\left|\sum_{j=1}^M \mathcal{E}^{cG(1)}_{K_j}\right| > TOL$
    \footnote{Use $\left|M(u^h) - M(u^{h-1})\right| > \gamma TOL$ for orthogonal
    error representation.}}{
      \nl For the mesh $\mathcal{T}_h$: compute the primal problem and the dual
            problem. \\
      \nl Mark some chosen percentage of the elements with highest
        $\mathcal{E}^{cG(1)}_{K_j}$ for refinement.\\
      \nl Generate the refined mesh $\mathcal{T}_{h+1}$, set $h = h + 1$
  }
\end{minipage}
\end{algorithm}


  \section{Results} \label{sec:Results}
  In this section we present a detailed computational study of the automated
goal-oriented adaptive algorithm presented in \autoref{sec:Algorithm} applied to
some time-dependent test problems. To this end we first we apply our algorithm
to a linear problem, \autoref{tst:Heat}, where the error representation is
orthogonal due to Galerkin orthogonality. Namely, we demonstrate the
effectiveness of \autoref{alg:Adaptivity} on the heat equation. Next we
demonstrate the effectiveness of \autoref{alg:Adaptivity} on non-linear
problems, where the error representation is no longer orthogonal. To this end,
we apply the algorithm to the benchmark problems presented in Sch\"afer et
al.\cite{Schaefer1996}, \autoref{tst:Cylinder2D}, \autoref{tst:Rectangle2D}, and
\autoref{tst:Rectangle3D}, where the equation of interest is the Icompressible
Navier-Stokes equation and our particular focus is on flow around bluff bodies.
Finally, in \autoref{tst:rhoNSE}, we apply our algorithm to the ever more
complex problem of Variable Density Incompressible Navier-Stokes equation, where
we focus on the solution to Rayleigh-Taylor instability.

For each test below we set the max iteration to $20$ and set the $TOL=0$. In
this way we will treat the solution obtained from the final adaptive mesh as the
so-called 'true solution' and use this solution to determine the effectivity of
the adaptive algorithm.

\subsection{Orthogonal Error Representation}
In this subsection we demonstrate the effectiveness of \autoref{alg:Adaptivity}
for problems where the global error representation is orthogonal. In these types
of problems the traditional thought is that the error representation doesn't
provide information about the error. However, in \autoref{tst:Heat}, we show
that the local error representation does, indeed, contain significant
information about the error and thus is an effective way of controlling the
error.

\begin{test}[Anisotropic Heat Equation] \label{tst:Heat}
  The heat equation is a purely parabolic linear partial differential equation
  and therefore the global error representation is zero due to Galerkin
  Orthogonality.  Thus it is an excellent choice to demonstrate the
  effectiveness of the algorithm described in \autoref{sec:Algorithm}.

  In this test problem we extend the test given by Li et al.
  \cite[Example 5.1]{Li2010}, i.e. the anisotropic diffusion equation, to the
  time dependent case. Thus we are concerned with the numerical solution of the
  anisotropic heat equation on a domain with a hole given by
  \begin{equation} \label{eq:Heat}
    \begin{split}
      u_t - \nabla \cdot (\kappa \nabla u) &= f, \quad \text{in } \Omega \\
      u &= g, \quad \text{on } \partial \Omega \\
      u(0;x,y) &= u_0(x,y)
    \end{split}
  \end{equation}
  where $\Omega = \Omega_1\setminus~\Omega_2$ is the physical domain, $f$ is the
  forcing function, $g$ is the function describing the boundary condition, and
  $\kappa$ is the diffusion tensor. For this test we take $\Omega_1 = [0,
  1]^2$, $\Omega_2 = [\frac{4}{9}, \frac{5}{9}]^2$, $f = 0$
  \begin{equation}
    g = \begin{cases}
        0 & \text{if } (x,y) \in \partial \Omega_1 \\
        0.5 - 0.5\, \cos(10\, t) & \text{if } (x,y) \in \partial \Omega_2,
    \end{cases}
    \label{eqn:BCFunction}
  \end{equation}
  and the diffusion tensor to be
  \begin{equation}
    \kappa = \begin{bmatrix} 500.5 & 499.5 \\ 499.5 & 500.5 \end{bmatrix}.
    \label{eqn:DiffusionTensor}
  \end{equation}
  The above diffusion tensor results in a diffusion process which is
  preferential along the line $x = y$.
  \textcolor{red}{Need info on discretization.}
\end{test}

\subsection{Non-Orthogonal Error Representation}
In this subsection we demonstrate the effectiveness of \autoref{alg:Adaptivity}
applied to problems where the global error representation is no longer
orthogonal. In particular, we look at applying \autoref{alg:Adaptivity} to the
Incompressible Navier-Stokes and the Variable Density Incompressible Navier
Stokes. In both cases, we solve each test case using a
Galerkin\slash~Least-Squares (GLS) stabilized cG(1)cG(1) finite element, where
the first cG(1) indicates that the test and trial functions are both piecewise
linear, while the second cG(1) indicates that the test piecewise constants and
the trial functions are continuous piecewise linear.

\subsubsection{Incompressible Navier-Stokes}
For test problems \ref{tst:Cylinder2D}, \ref{tst:Rectangle2D}, and
\ref{tst:Rectangle3D} we are concerned with the Incompressible Navier-Stokes
with constant kinematic viscosity, $\nu>0$, in a domain $\Omega\subset \R^d$,
with boundary $\partial \Omega$, i.e.
where the strong residual is given by
\begin{equation}
    \begin{split}
      \mathbf{u}_t + \left( \mathbf{u} \cdot \nabla \right) \mathbf{u} - \nu\,
          \Delta \mathbf{u} + \nabla p = \mathbf{f}, \quad \mathbf{x} \in \Omega \\
          \nabla \cdot \mathbf{u} = 0, \quad \mathbf{x} \in \Omega
    \end{split}
  \label{eqn:NSE}
\end{equation}
Given a step size $k$ and applying the cG(1)cG(1) finite element discretization
to \eqref{eqn:NSE} with $V = (v, q) \in X \subset [H^1_0(\Omega)]^d \times
H^1(\Omega)$ gives the following weak residual
\begin{equation}
  \begin{split}
    r(\bar{U}^n; V) &= \left(\mathbf{u}^n - \mathbf{u}^{n-1}\right)\,k^{-1}
        + (\left( \bar{\mathbf{u}}^n \cdot \nabla \right) \bar{\mathbf{u}}^n, v) \\
        &\quad- \nu\, (\nabla \bar{\mathbf{u}}^n, \nabla v)
        - (p^n, \nabla \cdot v) - (\mathbf{f}, v)
        + (\nabla \cdot \bar{\mathbf{u}}^n, v) = 0
  \end{split}
  \label{eqn:WeakNSE}
\end{equation}
where $\bar{U}^n = (\bar{\mathbf{u}}^n,p)$, and $\bar{\mathbf{u}}^n =
\frac{1}{2}\left(\mathbf{u}^n + \mathbf{u}^{n-1}\right)$. Given the solution
$U=(\mathbf{u},p)$, since the elements we are concerned with (cG(1)cG(1)) have
test functions which are linear in space and constant in time the strong
residual is given by
\begin{equation}
    R(\bar{U}^n,U) = \begin{cases}
      \left(\mathbf{u} \cdot \nabla \right) \bar{\mathbf{u}}^n
        + \nabla p^n - \mathbf{f} = 0 \\
      \nabla \cdot \bar{\mathbf{u}}^n = 0.
    \end{cases}
  \label{eqn:StrongNSE}
\end{equation}
\textcolor{red}{Should $\mathbf{f}$ be an average between time steps?}
Finally, the GLS-cG(1)cG(1) discretization of \eqref{eqn:NSE} is given by
\begin{equation}
  r(\bar{U}^n,V) + SD_{\delta}^n(\bar{U}^n,V) = 0, \quad \forall V=(v,q) \in X.
  \label{eqn:G2}
\end{equation}
Here $SD_{\delta}^n$ is the GLS and is given by
\begin{equation}
  SD_{\delta}^n \equiv
    \delta_1 (\left(\bar{\mathbf{u}}^n \cdot \nabla \right) \bar{\mathbf{u}}^n
        + \nabla p^n - \mathbf{f},
      \left(\bar{\mathbf{u}}^n \cdot \nabla \right) v + \nabla q)
      + \delta_2 (\nabla \cdot \bar{\mathbf{U}}^n, \nabla \cdot v),
  \label{eqn:NSEStabilization}
\end{equation}
where $\delta_1 = \kappa_1 (k^{-1} + |\mathbf{u}^{n-1}|^2\, h^{-2})^{-1/2}$, and
$\delta_2 = \kappa_2 h$, while $\kappa_1$ and $\kappa_2$ are problem independent
constants of unit size.


\begin{test}[Flow Around a Cylinder] \label{tst:Cylinder2D}
\end{test}

\begin{test}[2D Flow Around a Square] \label{tst:Rectangle2D}
\end{test}

\begin{test}[3D Flow Around a Rectangular-Cylinder] \label{tst:Rectangle3D}
\end{test}

\subsubsection{Variable Density Incompressible Navier-Stokes}
\begin{test}[Variable Density Navier-Stokes] \label{tst:rhoNSE}
\end{test}


  \section{Conclusions} \label{sec:Conclusions}
  In this report we presented a new approach to automate goal-oriented error
control to solutions of partial differential equations. We demonstrated the
effectiveness of this approach on linear and non-linear problems. In particular,
we show that the method is more effective when the chosen functional is a
measure of local phenomenon as opposed to a more global phenomenon, since the
adaptivity turns out to be similar to uniform refinement.



  \section*{Acknowledgements}
  This research has been supported by EU-FET grant EUNISON 308874, the European
  Research Council, the Swedish Foundation for Strategic Research, the Swedish
  Research Council, the Basque Excellence Research Center (BERC) program by
  the Basque Government, and the Severo Ochoa excellence award by the Spanish
  Government.

  \printbibliography
\end{document}
