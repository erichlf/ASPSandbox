Systems of \emph{Partial Differential Equations} (PDEs) describe the world we
live in, however obtaining accurate solutions to such systems can be quite
challenging. Thus, it is pivotal to be able to determine the quality of computed
solutions. Due to the challenging nature of automatically accessing the quality
of such solutions it is very typical for the quality to be accessed manually by
the scientist/engineer running the simulation. Not only is this approach
unreliable, but is also time consuming.

\emph{Goal oriented mesh adaptivity} has the potential to mitigate such
problems.  Goal oriented mesh adaptivity is based on an {\it a posteriori} error
estimates, where the error in a output quantity of interest (goal functional) is
minimized using duality techniques. The method of using {\it a posteriori} error
estimation using a duality argument has been developed since the 1970s and 1980s
(see \cite{Babuska1978, Babuska1981, Verfurth1989} for some examples).  In this
technique the adaptive algorithm will attempt to optimize the mesh so as to
minimize the error in some goal functional, thus minimizing the computational
cost while minimizing errors in the quantity of interest. Typically this is
accomplished by iteritively refining the mesh in areas where \emph{error
indicators}, which characterises the contribution of the global error to some
goal functional, are greatest. To obtain the error indicators one solves an
auxilary problem, based on the linear dual (adjoint) problem.

Although, adaptive finite elements are becoming more and more widespread
across many different fields \cite{Eriksson1995, Behrens1998, Becker2001,
Rognes2010, Hoffman2011, Izarra2014, Jansson2014a}, the simplicity and
generality of the method is still lacking. That is, the current state of goal
oriented adaptivity is relegated to experts, because an {\it a posteriori}
analysis must be performed on each problem. This {\it a posteriori} error
estimate takes the form of an inner product of a dual problem, which the
\emph{Finite Element Method} (FEM) is adapted well to. However, due to Galerkin
orthogonality the inner product representing the error vanishes, which seems to
suggest the global error is zero. However, this is clearly not the case. To
avoid this problem one can use one of two different standard approaches:
\begin{inparaenum}[(i)]
    \item \label{enu:Spaces} by taking the dual and primal problems to be
        in different spaces, or
    \item \label{enu:CS} by projecting the dual problem onto the finite
        element space and then using Cauchy-Schwarz inequality together with
        interpolation error estimates to bound the local error on each
        element.
\end{inparaenum}
Each of these methods present their own drawbacks and both present a lack of
simplicity and generality: (\ref{enu:Spaces}) results in increased computational
cost and therefore less efficient, and (\ref{enu:CS}) lacks sharpnesss and
therefore results in a method which is less efficiency and less reliable
\cite{Jansson2014b}.

In \cite{Jansson2014a, Jansson2014b} they present a novel approach to goal
oriented mesh adaptivity for stationary problems, which uses the {\it a
posteriori} error representation directly with the dual and primal problems in
the same space. The error representation has typically been considered to
contain no real information with respect to the {a posteriori} error, however
they show that, in fact, the error representation does contain valuable
information. Additionally, this approach avoids the drawbacks of both
(\ref{enu:Spaces}) and (\ref{enu:CS}), while offering a general and simple
method. Thereby, offering a method which is efficient and accessible to
non-experts.

It is our goal in this paper to extend the novel approach for stationary
problems developed in \cite{Jansson2014a, Jansson2014b} to time-dependent
problem, whereby goal oriented adaptivity can be automated, requiring minimal
user input, thus  saving scientists/engineers many hours of times. Therefore, we
will make goal oriented mesh adaptivity accessible by scientists/engineers who
are not experts in this area. To this end we present our algorithm in
\autoref{sec:Algorithm}, and then we present a few numerical experiments in
\autoref{sec:Results}, based in DOLFIN\cite{Alnae2011} and using
DOLFIN-Adjoint\cite{Ham2012} for the automated calculation of the adjoint
problem, to demonstrate the effectiveness of the algorithm. Finally, we make our
concluding remarks and observations in \autoref{sec:Conclusions}.
