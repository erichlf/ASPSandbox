In this section we develop the theoretical framework for the goal oriented mesh
adaptivity, including the space-time finite element formulation, \emph{a priori}
error estimation, the dual problem, \emph{a posteriori} error estimation,
adaptive error control, and the error estimate. To this end we begin by
describing a general residual based space-time finite element discretization for
the linear variational problem.

We initially assume a general linear time-dependent variational problem and
develop a space-time finite element discretization based on the weak residual.
To this end we let
\begin{equation*}
    I = \{0\} \cup I_1 \cup I_2 \cup \cdots \cup I_{M-1} \cup I_M
\end{equation*}
be a partition of the time interval $I = [0, T]$ with $I_n = (t_{n-1}, t_n]$,
where $0=t_0 < t_1 < \ldots < t_N = T$, and step size, $k_n = t_n - t_{n-1}$.
Furthermore, for each $n$ let $\mathcal{T}_n= \{K\}$ be a triangulation of the
spacial domain, $\Omega$, then we can descretize our domain into space-time
slabs $S^h_n = \mathcal{T}_n \times I_n$. Taking $W^h_n = V^h_n \times Q_n$ an
appropriate space-time finite element discretization of $S^h_n$, with $h_n(x)$
the mesh size associated with the triangulation $\mathcal{T}_n$. Then the finite
element method in weak-residual form is given by
\begin{equation}
    \begin{split}
        &\text{Find } w^h_n \in W^h_n \text{ such that} \\
        r(w^h_n, v^h) &:= (\dot{w}^h_n, v^h)
            + a(w^h_n, v^h) - L(v^h) = 0 \quad \forall v^h \in W^h_n,
    \end{split}
    \label{eq:WeakResidual}
\end{equation}
where $\dot{w}^h_n$ indicates the time derivative of the discrete
trial function $w^h_n$, $a(\cdot, \cdot)$ and $L(v)$ are the bilinear and linear
forms, respectively, associated with the variational problem, while $r(\cdot,
\cdot)$ is the weak-residual.

Now we can describe dual problem associated with the general residual based
space-time finite element introduced previously, \eqref{eq:WeakResidual}.
Namely, we introduce the discrete dual solution $\Phi^h_n$ associated with the
problem \eqref{eq:WeakResidual}. To this end, given the primal problem,
\eqref{eq:WeakResidual}, and the goal functional $\psi$ we the dual solution is
given by:
\begin{equation*}
    \begin{split}
        &\text{Find } \Phi^h_n \in (W^h_n)^* \text{ such that} \\
        r^*(\Phi^h_n, v^h) := r(v^h, \Phi^h_n) &= (\dot{v}^h, \Phi^h_n)
            + a(v^h, \Phi^h_n) - (v^h, \psi_n) = 0 \quad
            \forall v^h \in W^h_n,
    \end{split}
\end{equation*}
where $\psi$ is the projection of the goal functional $\mathcal{M}(\cdot) =
(\cdot, \psi)$ onto the finite element space $W^h_n$. In other words, given the
solution $w^h_n$ to \eqref{eq:WeakResidual} we solve the problem
\begin{equation}
    \begin{split}
        &\text{Find } \Phi^h_n \in (W^h_n)^* \text{ such that} \\
        r(v^h, \Phi^h_n) &:= (\dot{v}^h, \Phi^h_n)
            + a(v^h, \Phi^h_n) - \mathcal{M}(w^h_n) = 0 \quad \forall v^h \in W^h_n.
    \end{split}
    \label{eq:Dual}
\end{equation}

Given the solutions $w,\, w^h_n$ to the continuous and discrete primal problems,
respectively, while $\Phi, \Phi^h_n$ are the continuous and discrete dual
solutions, respectively, then taking $v$ and $v^h$ to be $w$ and $w^h_n$,
respectively, we can determine the error in the goal functional at time $t$ in
the following manner
\begin{align}
    \mathcal{M}(w) - \mathcal{M}(w^h_n) &= \left((\dot{w}, \Phi) + a(w, \Phi)\right)
                      - \left( (\dot{w}^h_n, \Phi) + a(w^h_n, \Phi) \right) \nonumber \\
    \mathcal{M}(w - w^h_n) &= (\dot{w} - \dot{w}^h_n, \Phi) + a(w - w^h_n, \Phi) \nonumber \\
    \mathcal{M}(e) &= 0 \label{eq:DualError}
\end{align}
where the last line, \eqref{eq:DualError} is due to Galerkin orthogonality. On
the other hand we have
\begin{align}
    \mathcal{M}(e) & = \left((\dot{w}, \Phi) + a(w, \Phi) - L(\Phi)\right)
             - \left((\dot{w}^h_n, \Phi) + a(w^h_n, \Phi) - L(\Phi) \right) \nonumber \\
    &= -\left((\dot{w}^h_n, \Phi) + a(w^h_n, \Phi) - L(\Phi) \right) \nonumber \\
    \mathcal{M}(e) &= -r(w^h_n, \Phi) \label{eq:ErrorRepresentation} \\
\end{align}

However, we must consider the contribution from each indiviudal element to
the total error. To this end we multiply the the weak residual, $r(\cdot,
\cdot)$, by a discontinuous constant $\Theta_{K_j}$ where
\begin{equation}
    \Theta_K = \begin{cases}
        1 & \text{if } x \in K \\
        0 & \text{otherwise}
    \end{cases}
    \label{eq:DGTest}
\end{equation}
allowing for the error to be partitioned into its contribution from each element
$K_j$.
Therefore, we see that the error at time $t$ is given by
\begin{equation}
    \mathcal{M}(e) = \sum_{j=1}^M r(w^h_n, \Phi) \, \Theta_{K_j},
         = \sum_{j=1}^M r(w^h_n, \Phi\, \Theta_{K_j}).
    \label{eq:ErrorSum}
\end{equation}
The rightmost equality being a result of linearity in the test function of
$r(\cdot, \cdot)$.  For simplicity we will write
\begin{equation*}
    r(w^h_n, \Phi\, \Theta_{K_j}) = r_{K_j}(w^h_n, \Phi)
\end{equation*}
and therefore \eqref{eq:ErrorSum} becomes
\begin{equation}
    \mathcal{M}(e) = \sum_{j=1}^M r_{K_j}(w^h_n, \Phi).
    \label{eq:TotalError}
\end{equation}
To determine the error over time we then integrate over the time interval $I$
\begin{equation*}
    \int_I\! \mathcal{M}(e)\, dt = \int_I\! r_{K_j}(w^h_n, \Phi)\, dt.
\end{equation*}
Finally, given a time step $k$ and using a trapezoid rule we can approximate the
time integration and determine a computable total error
\begin{equation}
    \int_I\! \mathcal{M}(e)\, dt = \frac{1}{k} \sum_{n=0}^N \sum_{j=1}^M r_{K_j}(w^h_n, \Phi).
    \label{eq:TotalErrorEstimate}
\end{equation}

Thus, if we want to control the total error after all time step in our
functional using adaptive mesh refinement we would want
\begin{equation}
    \begin{split}
        \int_I\! \mathcal{M}(e)\, dt &= \int_I\! r(w^h_n, \Phi)\, dt \\
            &= \int_I\! \sum_{j=1}^M r_{K_j}(w^h_n, \Phi)\, dt \\
            &= \frac{1}{k} \sum_{n=0}^N \sum_{j=1}^M r_{K_j}(w^h_n, \Phi) < TOL.
    \end{split}
    \label{eq:Tol}
\end{equation}
Thus we have an error estimate, \eqref{eq:TotalErrorEstimate}, and it appears
that determining the error in a goal functional, $\mathcal{M}(e)$, is equivalent to
evaluating the weak residual, \eqref{eq:WeakResidual}, using the primal and dual
solutions.

However, due to Galerkin orthogonality when taking the dual solution to be
the discrete dual $\Phi^h_n$ we would still see that
\begin{equation*}
    \frac{1}{k} \sum_{n=0}^N \sum_{j=1}^M r_{K_j}(w^h_n, \Phi^h_n) = 0.
\end{equation*}
While this seems impractical we see that the per element error is still useful
for indicating where mesh refinement should take place. On the other hand for
non-Galerkin orthogonal discretizations/problems, such as the $cG(1)cG(1)-GLS$
discretizations.

Taking $\mathcal{E}$ to be the cellwise error indicator at the end of our time
interval $I$, while taking $\mathcal{E}_{K_j, n}$ to be
the error indicator for cell $K_j$ at time $t_n$ then our cellwise error
indicator is given by
\begin{equation}
    \begin{split}
        \mathcal{E}_{K_j} &= \frac{1}{k} \sum_{n=0}^N r_{K_j}(w^h_n, \Phi^h_n) \\
                &= \frac{1}{k} \sum_{n=0}^N \mathcal{E}_{K_j, n}
    \end{split}
    \label{eq:CellErrorIndicator}
\end{equation}
while for non-Galerkin orthogonal discretization/problems the error indicator is
given by
\begin{equation}
    \begin{split}
        \mathcal{E} &= \frac{1}{k} \sum_{n=0}^N \sum_{j=0}^M r_{K_j}(w^h_n, \Phi^h_n) \\
            &= \frac{1}{k} \sum_{n=0}^N \sum_{j=0}^M  \mathcal{E}_{K_j, n}.
    \end{split}
    \label{eq:ErrorIndicator}
\end{equation}
To determine a stopping criterion for orthogonal discretizations/problems we
will have to look at the difference in goal functions from one mesh
discretization to another, i.e.
\begin{equation}
     \left|\mathcal{M}(w^{h_m}_n) - \mathcal{M}(w^{h_{m-1}}_n)\right|
        < \gamma\, TOL,
    \label{eq:OrthogonalError}
\end{equation}
where $\gamma>0$ is a problem specific constant.

With this in place our goal oriented adaptive algorithm is then given by
\autoref{alg:Adaptivity}.
\begin{algorithm}[!htp]
  \caption{Adaptive mesh refinement} \label{alg:Adaptivity}
\begin{minipage}[t]{\columnwidth} %this minipage is a hack to use footenotes
  \KwData{Let $(\mathcal{T}_m, W^{h_m}_n,\mathcal{P})$ be a finite element
    discretization of
    \begin{equation*}
        r(w^{h_m}_n ,v^{h_m}_n) = (\dot{w}^{h_m}_n, v^{h_m}_n)
            + a(w^{h_m}_n, v^{h_m}_n) - L(v^{h_m}_n).
    \end{equation*}}
  \SetAlgoLined
  \While{ $\left|\frac{1}{k} \sum\limits_{n=0}^{N} \sum\limits_{j=1}^M
        \mathcal{E}^{cG(1)}_{K_j, n}\right| > TOL$
      \footnote{Use $\left|\mathcal{M}(w^{h_m}_n) -
                \mathcal{M}(w^{h_{m-1}}_n)\right| > \gamma\, TOL$ for
        orthogonal error representation.}}{
      \nl For the mesh $\mathcal{T}$: compute the primal problem and the dual
            problem. \\
      \nl Mark some chosen percentage of the elements with highest
        $\frac{1}{k} \sum\limits_{n=0}^N \sum\limits_{j=1}^M
        \mathcal{E}^{cG(1)}_{K_j, n}$ for refinement.\\
      \nl Generate the refined mesh $\mathcal{T}_{m+1}$, set $m = m + 1$
  }
\end{minipage}
\end{algorithm}

\subsection{Stabilized Methods}
For stabilized methods, such as the $cG(1)cG(1)-GLS$ method presented in
\autoref{sec:Results}, we no-longer have Galerkin orthogonality and thus we can
say a bit more about the error in our goal functional, $\mathcal{M}$.

First, we introduce a general stabilized finite element discretization; use
the same space-time finite element discretization presented for
\eqref{eq:WeakResidual} then we can write a general stabilized finite element
discretization as the sum of a weak residual and stabilization terms
\begin{equation}
    r(w^h_n, v^h) + S_{\delta}(w^h_n, v^h) = 0,
    \label{eq:StableWeakResidual}
\end{equation}
where $r(w^h_n, v^h)$ is the discretized weak residual, $S_{\delta}(w^h_n, v^h)$
is the stabilization and $\delta$ is a stabilization parameter depending on the
mesh size, $h$. With the stabilization parameter depending on the mesh size,
$h$, one can control the amount of stabilization used in various regions of the
computational domain. This allows one to minimize the error, through
mesh refinement, associated to the stabilization, where it contributes most to a
goal functional.

To do this we will solve both the primal and dual problems,
\eqref{eq:WeakResidual} and \eqref{eq:Dual}, respectively, with the
stabilization terms present, however when calculating the error indicator
$\mathcal{E}^{S_{\delta}}$ we will eliminate the stabilization terms, i.e.
\begin{equation}
    \begin{split}
        \mathcal{E}^{S_{\delta}}
        &= \frac{1}{k} \sum_{n=0}^N \sum_{j=0}^M  \mathcal{E}^{S_{\delta}}_{K_j, n} \\
        &= \frac{1}{k} \sum_{n=0}^N \sum_{j=0}^M r_{K_j}(w^h_n, \Phi^h_n) \\
        &= \frac{1}{k} \sum_{n=0}^N r(w^h_n, \Phi^h_n) \\
        &= -\frac{1}{k} \sum_{n=0}^N S_{\delta}(w^h_n, \Phi^h_n)
    \end{split}
    \label{eq:StableErrorIndicator}
\end{equation}
