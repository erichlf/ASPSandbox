In this section we develop the theoretical framework for the goal oriented mesh
adaptivity, including the space-time finite element formulation, \emph{a priori}
error estimation, the dual problem, \emph{a posteriori} error estimation,
adaptive error control, and the error estimate. To this end we begin by
describing a general residual based space-time finite element discretization for
the linear variational problem.

We initially assume a general linear time-dependent variational problem and
develop a space-time finite element discretization based on the weak residual.
To this end we let
\begin{equation*}
    I = \{0\} \cup I_1 \cup I_2 \cup \cdots \cup I_{M-1} \cup I_M
\end{equation*}
be a partition of the time interval $I = [0, T]$ with $I_n = (t_{n-1}, t_n]$,
where $0=t_0 < t_1 < \ldots < t_N = T$, and step size, $k_n = t_n - t_{n-1}$.
Then we can descretize our domain into space-time slabs $S_n = \Omega \times
I_n$. Furthermore, for each $n$ let $\mathcal{T}_n= \{K\}$ be a triangulation
of the spacial domain, $\Omega$, then we take $W_n^h$ an appropriate finite
element space with $h_n(x)$ the mesh size associated with the triangulation
$\mathcal{T}_n$. Then the finite element method in weak-residual form is given
by
\begin{equation}
    \begin{split}
        &\text{Find } w^h \in W^h_n \text{ such that} \\
        r(w^h, v^h) &:= (w^h_t, v^h) + a(w^h, v^h) - L(v^h) = 0 \quad \forall
            v^h \in W^h_n,
    \end{split}
    \label{eq:WeakResidual}
\end{equation}
where $a(\cdot, \cdot)$ and $L(v)$ are the bilinear and linear forms,
respectively, associated with the variational problem, while $r(\cdot, \cdot)$
is the weak-residual.

Now we can describe a general dual problem associated with the general residual
based space-time finite element introduced previously, \eqref{eq:WeakResidual}.
Namely, we introduce the dual solution $\Phi$ associated with the problem
\eqref{eq:WeakResidual}. To this end, given the primal problem,
\eqref{eq:WeakResidual}, then we assume the dual solution is given by:
\begin{equation*}
    \begin{split}
        &\text{Find } \Phi \in (W^h_n)^* \text{ such that} \\
        r^*(\Phi, v^h) := r(v^h, \Phi) &= (v^h_t, \Phi) + a(v^h, \Phi)
            - (v^h, \psi) = 0 \quad \forall v^h \in W^h_n,
    \end{split}
\end{equation*}
where $\psi$ is the projection of the goal functional $M(\cdot) = (\cdot, \psi)$
onto the finite element space $W^h_n$. In other words, given the solution $w^h$
to \eqref{eq:WeakResidual} we solve the problem
\begin{equation}
    \begin{split}
        &\text{Find } \Phi \in (W^h_n)^* \text{ such that} \\
        r(v^h, \Phi) &:= (v^h_t, \Phi) + a(v^h, \Phi) - M(w^h) = 0
                       \quad \forall v^h \in W^h.
    \end{split}
    \label{eq:Dual}
\end{equation}

Given the solutions $w,\, w^h$ to the primal and discrete primal problems,
respectively, while $\Phi$ is the solution to the dual problem then
taking $v$ and $v^h$ to be $w$ and $w^h$, respectively, we can determine the
error in the goal functional at time $t$ in the following manner
\begin{align}
    M(w) - M(w^h) &= \left((w_t, \Phi) + a(w, \Phi)\right)
                      - \left( (w^h_t, \Phi) + a(w^h, \Phi) \right) \nonumber \\
    M(w - w^h) &= (w_t - w^h_t, \Phi) + a(w - w^h, \Phi) \nonumber \\
    M(e) &= 0 \label{eq:DualError}
\end{align}
where the last line, \eqref{eq:DualError} is due to Galerkin orthogonality. On
the other hand we have
\begin{align}
    M(e) & = \left((w_t, \Phi) + a(w, \Phi) - L(\Phi)\right)
             - \left((w^h_t, \Phi) + a(w^h, \Phi) - L(\Phi) \right) \nonumber \\
    &= -\left((w^h_t, \Phi) + a(w^h, \Phi) - L(\Phi) \right) \nonumber \\
    M(e) &= -r(w^h, \Phi) \label{eq:ErrorRepresentation} \\
\end{align}

However, we must consider the contribution from each indiviudal element to
the total error. To this end we multiply the the weak residual, $r(\cdot,
\cdot)$, by a discontinuous constant $\Theta_{K_j}$ where
\begin{equation}
    \Theta_K = \begin{cases}
        1 & \text{if } x \in K \\
        0 & \text{otherwise}
    \end{cases}
    \label{eq:DGTest}
\end{equation}
allowing for the error to be partitioned into its contribution from each element
$K_j$.
Therefore, we see that the error at time $t$ is given by
\begin{equation}
    M(e) = \sum_{j=1}^M r(w^h, \Phi) \, \Theta_{K_j},
         = \sum_{j=1}^M r(w^h, \Phi\, \Theta_{K_j}).
    \label{eq:ErrorSum}
\end{equation}
The rightmost equality being a result of linearity in the test function of
$r(\cdot, \cdot)$.  For simplicity we take
\begin{equation*}
    r(w^h, \Phi\, \Theta_{K_j}) = r_{K_j}(w^h, \Phi)
\end{equation*}
and therefore \eqref{eq:ErrorSum} becomes
\begin{equation}
    M(e) = \sum_{j=1}^M r_{K_j}(w^h, \Phi).
    \label{eq:TotalError}
\end{equation}
To determine the error over time we then integrate over the time interval $I$
\begin{equation*}
    \int_I\! M(e)\, dt = \int_I\! r_{K_j}(w^h, \Phi)\, dt.
\end{equation*}
Finally, given a time step $k$ and using a trapezoid rule we can approximate the
time integration and determine a computable total error
\begin{equation}
    \int_I\! M(e)\, dt = \frac{1}{k} \sum_{n=0}^N \sum_{j=1}^M r_{K_j}(w^h, \Phi).
    \label{eq:TotalErrorEstimate}
\end{equation}

Thus, if we want to control the total error after all time step in our
functional using adaptive mesh refinement we would want
\begin{equation}
    \begin{split}
        \int_I\! M(e)\, dt &= \int_I\! r(w^h, \Phi)\, dt \\
            &= \int_I\! \sum_{j=1}^M r_{K_j}(w^h, \Phi)\, dt \\
            &= \frac{1}{k} \sum_{n=0}^N \sum_{j=1}^M r_{K_j}(w^h, \Phi) < TOL.
    \end{split}
    \label{eq:Tol}
\end{equation}
we have an error estimate, \eqref{eq:TotalErrorEstimate}, and it appears
that determining the error in a goal functional, $M(e)$, is equivalent to
evaluating the weak residual, \eqref{eq:WeakResidual}, using the primal and dual
solutions.

Taking $\mathcal{E}$ to be the cellwise error indicator at the end of our time
interval $I$, while taking $\mathcal{E}_{K_j, n}$ to be
the error indicator for cell $K_j$ at time $t_n$ then our error indicator is
given by
\begin{equation}
    \begin{split}
    \mathcal{E}(x) &= \frac{1}{k} \sum_{n=0}^N \sum_{j=0}^M r_{K_j}(w^h, \Phi) \\
                &= \frac{1}{k} \sum_{n=0}^N \sum_{j=0}^M \mathcal{E}_{K_j, n}(x)
    \end{split}
    \label{eq:ErrorIndicator}
\end{equation}

With this in place our goal oriented adaptive algorithm is then given by
\autoref{alg:Adaptivity}.
\begin{algorithm}[!htp]
  \caption{Adaptive mesh refinement} \label{alg:Adaptivity}
\begin{minipage}[t]{\columnwidth} %this minipage is a hack to use footenotes
  \KwData{Let $(\mathcal{T}_m, W^{h_m}_n,\mathcal{P})$ be a finite element
    discretization of
    \begin{equation*}
        r(w^{h_m} ,v^{h_m}) = (w^{h_m}_t, v^{h_m}) + a(w^{h_m}, v^{h_m}) -
            L(v^{h_m}).
    \end{equation*}}
  \SetAlgoLined
  \While{ $\left|\frac{1}{k} \sum\limits_{n=0}^{N} \sum\limits_{j=1}^M
        \mathcal{E}^{cG(1)}_{K_j, n}\right| > TOL$
      \footnote{Use $\left|M(w^{h_m}) - M(w^{h_{m-1}})\right| > \gamma TOL$ for
        orthogonal error representation.}}{
      \nl For the mesh $\mathcal{T}$: compute the primal problem and the dual
            problem. \\
      \nl Mark some chosen percentage of the elements with highest
        $\frac{1}{k} \sum\limits_{n=0}^N \sum\limits_{j=1}^M
        \mathcal{E}^{cG(1)}_{K_j, n}$ for refinement.\\
      \nl Generate the refined mesh $\mathcal{T}_{m+1}$, set $m = m + 1$
  }
\end{minipage}
\end{algorithm}
