In this section we initially assume a general linear variational problem of the
form:
\begin{equation}
    \begin{split}
        &\text{find } u \in V \text{ such that} \\
        r(u, v) &:= a(u, v) - L(v) = 0 \quad \forall v \in V,
    \end{split}
    \label{eq:WeakResidual}
\end{equation}
where $a(\cdot, \cdot)$ and $L(v)$ are the bilinear and linear forms associated
with the variational problem, while $r(\cdot, \cdot)$ is the weak-residual.
While we assume the forms are linear the extension to non-linear problems is
quite straightforward.

Additionally, given the primal problem, \eqref{eq:WeakResidual}, then we assume
the dual solution is given by:
\begin{equation*}
    \begin{split}
        &\text{find } \Phi \in V^* \text{ such that} \\
        r(v, \Phi) &:= a(v, \Phi) - (v, \psi) = 0 \quad \forall v \in V,
    \end{split}
\end{equation*}
where $\psi$ is the projection of the goal functional $M(\cdot) = (\cdot, \psi)$
onto the finite element space $V$. In other words, given the solution $u$ to
\eqref{eq:WeakResidual} we solve the problem
\begin{equation}
    \begin{split}
        &\text{find } \Phi \in V^* \text{ such that} \\
        r(v, \Phi) &:= a(v, \Phi) - M(u) = 0 \quad \forall v \in V.
    \end{split}
    \label{eq:Dual}
\end{equation}

Given the solution $u$ to the primal problem and $u^h$ the solution to the
discrete primal problem, while $\Phi$ is the solution to the dual problem then
taking $v$ and $v^h$ to be $u$ and $u^h$, respectively, we can determine the
error in the goal functional in the following manner
\begin{align}
    M(u) - M(u^h) &= a(u, \Phi) - a(u^h, \Phi) \nonumber \\
    M(u - u^h) &= a(u - u^h, \Phi) \nonumber \\
    M(e) &= 0 \label{eq:DualError}
\end{align}
where the last line, \eqref{eq:DualError} is due to Galerkin orthogonality. On
the other hand we see
\begin{align}
    M(e) & = \cancelto{0}{a(u, \Phi) - L(\Phi)}
             - \left( a(u^h, \Phi) - L(\Phi) \right) \nonumber \\
    &= -r(u^h, \Phi). \label{eq:ErrorRepresentation}
\end{align}
Thus, it appears that the determining the error in a goal functional, $M(e)$, is
equivalent to evaluating the weak residual, \eqref{eq:WeakResidual}, using the
primal and dual solutions. However, due to Galerkin orthogonality the error
representation, \eqref{eq:ErrorRepresntation} evaluates to zeo if the
dual solution, $\Phi$, and the primal solution, $u^h$, are in the same space.

\begin{algorithm}[!htp]
\begin{minipage}[t]{\columnwidth} %this minipage is a hack to use footenotes
  \caption{Adaptive mesh refinement} \label{alg:Adaptivity}
  \KwData{Let $(\mathcal{T}_k, V_k,\mathcal{P})$ be a finite element
    discretization of $r(U,v) = a(U,v) - L(v)$.} \vspace{0.5em}
  \SetAlgoLined
  \While{ $\left|\sum_{j=1}^M \mathcal{E}^{cG(1)}_{K_j}\right| > TOL$
    \footnote{Use $\left|M(U^k) - M(U^{k-1}\right| > \gamma TOL$ for orthogonal
    error representation.}}{
      \nl For the mesh $\mathcal{T}_k$: compute the primal problem and the dual
            problem. \\
      \nl Mark some chosen percentage of the elements with highest
        $\mathcal{E}^{cG(1)}_{K_j}$ for refinement.\\
      \nl Generate the refined mesh $\mathcal{T}_{k+1}$, set $k = k + 1$
  }
\end{minipage}
\end{algorithm}
