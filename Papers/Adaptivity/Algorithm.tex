In this section we initially assume a general linear time-dependent variational
problem of the form:
\begin{equation}
    \begin{split}
        &\text{find } u \in V \text{ such that} \\
        r(u, v) &:= (u_t, v) + a(u, v) - L(v) = 0 \quad \forall v \in V,
    \end{split}
    \label{eq:WeakResidual}
\end{equation}
where $a(\cdot, \cdot)$ and $L(v)$ are the bilinear and linear forms associated
with the variational problem, while $r(\cdot, \cdot)$ is the weak-residual.

Additionally, given the primal problem, \eqref{eq:WeakResidual}, then we assume
the dual solution is given by:
\begin{equation*}
    \begin{split}
        &\text{find } \Phi \in V^* \text{ such that} \\
        r(v, \Phi) &:= (u_t, \Phi) + a(v, \Phi) - (v, \psi) = 0
                       \quad \forall v \in V,
    \end{split}
\end{equation*}
where $\psi$ is the projection of the goal functional $M(\cdot) = (\cdot, \psi)$
onto the finite element space $V$. In other words, given the solution $u$ to
\eqref{eq:WeakResidual} we solve the problem
\begin{equation}
    \begin{split}
        &\text{find } \Phi \in V^* \text{ such that} \\
        r(v, \Phi) &:= (u_t, \Phi) + a(v, \Phi) - M(u) = 0
                       \quad \forall v \in V.
    \end{split}
    \label{eq:Dual}
\end{equation}

Given the solutions $u,\, u^h$ to the primal and discrete primal problems,
respectively, while $\Phi$ is the solution to the dual problem then
taking $v$ and $v^h$ to be $u$ and $u^h$, respectively, we can determine the
error in the goal functional in the following manner
\begin{align}
    M(u) - M(u^h) &= \left((u_t, \Phi) + a(u, \Phi)\right)
                      - \left( (u^h_t, \Phi) + a(u^h, \Phi) \right) \nonumber \\
    M(u - u^h) &= (u_t - u^h_t, \Phi) + a(u - u^h, \Phi) \nonumber \\
    M(e) &= 0 \label{eq:DualError}
\end{align}
where the last line, \eqref{eq:DualError} is due to Galerkin orthogonality. On
the other hand we have
\begin{align}
    M(e) & = \left((u_t, \Phi) + a(u, \Phi) - L(\Phi)\right)
             - \left((u^h_t, \Phi) + a(u^h, \Phi) - L(\Phi) \right) \nonumber \\
    &= -\left((u^h_t, \Phi) + a(u^h, \Phi) - L(\Phi) \right) \nonumber \\
    &= -r(u^h, \Phi) \label{eq:ErrorRepresentation} \\
    |M(e)| &= |r(u^h, \Phi)| \label{eq:ErrorEstimate}
\end{align}
Thus, we have an error estimate, \eqref{eq:ErrorEstimate}, and it appears that
the determining the error in a goal functional, $M(e)$, is equivalent to
evaluating the weak residual, \eqref{eq:WeakResidual}, using the primal and dual
solutions. However, due to Galerkin orthogonality the error representation,
\eqref{eq:ErrorRepresentation} evaluates to zero if the dual solution, $\Phi$,
and the primal solution, $u^h$, are in the same space.

\begin{algorithm}[!htp]
\begin{minipage}[t]{\columnwidth} %this minipage is a hack to use footenotes
  \caption{Adaptive mesh refinement} \label{alg:Adaptivity}
  \KwData{Let $(\mathcal{T}_h, V^h,\mathcal{P})$ be a finite element
    discretization of $r(u^h ,v^h) = (u^h_t, v^h) + a(u^h, v^h) - L(v^h)$.}
    \vspace{0.5em}
  \SetAlgoLined
  \While{ $\left|\sum_{j=1}^M \mathcal{E}^{cG(1)}_{K_j}\right| > TOL$
    \footnote{Use $\left|M(u^h) - M(u^{h-1})\right| > \gamma TOL$ for orthogonal
    error representation.}}{
      \nl For the mesh $\mathcal{T}_h$: compute the primal problem and the dual
            problem. \\
      \nl Mark some chosen percentage of the elements with highest
        $\mathcal{E}^{cG(1)}_{K_j}$ for refinement.\\
      \nl Generate the refined mesh $\mathcal{T}_{h+1}$, set $h = h + 1$
  }
\end{minipage}
\end{algorithm}
