In this section we develop the theoretical framework for the goal oriented mesh
adaptivity, including the space-time finite element formulation, \emph{a priori}
error estimation, the dual problem, \emph{a posteriori} error estimation,
adaptive error control, and the error estimate. To this end we begin by
describing a general residual based space-time finite element discretization for
the linear variational problem.

\subsection{Residual Based Space-Time Finite Elements} \label{sse:STFE}
In this sub-section we initially assume a general linear time-dependent
variational problem. Let
\begin{equation*}
    I = \{0\} \cup I_1 \cup I_2 \cup \cdots \cup I_{M-1} \cup I_M
\end{equation*}
be a partition of the time interval $I = [0, T]$ with $I_n = (t_{n-1}, t_n]$,
where $0=t_0 < t_1 < \ldots < t_N = T$, and step size, $k_n = t_n - t_{n-1}$.
Then we can descretize our domain into space-time slabs $S_n = \Omega \times
I_n$. Furthermore, for each $n$ let $\mathcal{T}_n= \{K\}$ be a triangulation
of the spacial domain, $\Omega$, then we take $W_n^h$ an appropriate finite
element space with $h_n(x)$ the mesh size associated with the triangulation
$\mathcal{T}_n$. Then the finite element method in weak-residual form is given
by
\begin{equation}
    \begin{split}
        &\text{Find } w^h \in W^h_n \text{ such that} \\
        r(w^h, v^h) &:= (w^h_t, v^h) + a(w^h, v^h) - L(v^h) = 0 \quad \forall
            v^h \in W^h_n,
    \end{split}
    \label{eq:WeakResidual}
\end{equation}
where $a(\cdot, \cdot)$ and $L(v)$ are the bilinear and linear forms,
respectively, associated with the variational problem, while $r(\cdot, \cdot)$
is the weak-residual.

\subsection{Dual Problem} \label{sse:Dual}
In this sub-section we describe a general dual problem associated with the
general residual based space-time finite element introduced in the previous
subsection. Namely, we introduce the dual solution $\Phi$ associated with the
problem \eqref{eq:WeakResidual}. To this end, given the primal problem,
\eqref{eq:WeakResidual}, then we assume the dual solution is given by:
\begin{equation*}
    \begin{split}
        &\text{Find } \Phi \in (W^h_n)^* \text{ such that} \\
        r^*(\Phi, v^h) := r(v^h, \Phi) &= (v^h_t, \Phi) + a(v^h, \Phi)
            - (v^h, \psi) = 0 \quad \forall v^h \in W^h_n,
    \end{split}
\end{equation*}
where $\psi$ is the projection of the goal functional $M(\cdot) = (\cdot, \psi)$
onto the finite element space $W^h_n$. In other words, given the solution $w^h$
to \eqref{eq:WeakResidual} we solve the problem
\begin{equation}
    \begin{split}
        &\text{Find } \Phi \in (W^h_n)^* \text{ such that} \\
        r(v^h, \Phi) &:= (v^h_t, \Phi) + a(v^h, \Phi) - M(w^h) = 0
                       \quad \forall v^h \in W^h.
    \end{split}
    \label{eq:Dual}
\end{equation}

Given the solutions $w,\, w^h$ to the primal and discrete primal problems,
respectively, while $\Phi$ is the solution to the dual problem then
taking $v$ and $v^h$ to be $w$ and $w^h$, respectively, we can determine the
error in the goal functional in the following manner
\begin{align}
    M(w) - M(w^h) &= \left((w_t, \Phi) + a(w, \Phi)\right)
                      - \left( (w^h_t, \Phi) + a(w^h, \Phi) \right) \nonumber \\
    M(w - w^h) &= (w_t - w^h_t, \Phi) + a(w - w^h, \Phi) \nonumber \\
    M(e) &= 0 \label{eq:DualError}
\end{align}
where the last line, \eqref{eq:DualError} is due to Galerkin orthogonality. On
the other hand we have
\begin{align}
    M(e) & = \left((w_t, \Phi) + a(w, \Phi) - L(\Phi)\right)
             - \left((w^h_t, \Phi) + a(w^h, \Phi) - L(\Phi) \right) \nonumber \\
    &= -\left((w^h_t, \Phi) + a(w^h, \Phi) - L(\Phi) \right) \nonumber \\
    &= -r(w^h, \Phi) \label{eq:ErrorRepresentation} \\
    |M(e)| &= |r(w^h, \Phi)| \label{eq:ErrorEstimate}
\end{align}
Thus, we have an error estimate, \eqref{eq:ErrorEstimate}, and it appears that
determining the error in a goal functional, $M(e)$, is equivalent to
evaluating the weak residual, \eqref{eq:WeakResidual}, using the primal and dual
solutions.

\begin{algorithm}[!htp]
\begin{minipage}[t]{\columnwidth} %this minipage is a hack to use footenotes
  \caption{Adaptive mesh refinement} \label{alg:Adaptivity}
  \KwData{Let $(\mathcal{T}_m, W^{h_m}_n,\mathcal{P})$ be a finite element
    discretization of
    \begin{equation*}
        r(w^{h_m} ,v^{h_m}) = (w^{h_m}_t, v^{h_m}) + a(w^{h_m}, v^{h_m}) -
            L(v^{h_m}).
    \end{equation*}}
  \SetAlgoLined
  \While{ $\left|\frac{1}{k} \sum\limits_{n=0}^{N} \sum\limits_{j=1}^M
        \mathcal{E}^{cG(1)}_{K_j, n}\right| > TOL$
      \footnote{Use $\left|M(w^{h_m}) - M(w^{h_{m-1}})\right| > \gamma TOL$ for
        orthogonal error representation.}}{
      \nl For the mesh $\mathcal{T}$: compute the primal problem and the dual
            problem. \\
      \nl Mark some chosen percentage of the elements with highest
        $\frac{1}{k} \sum\limits_{n=0}^N \sum\limits_{j=1}^M
        \mathcal{E}^{cG(1)}_{K_j, n}$ for refinement.\\
      \nl Generate the refined mesh $\mathcal{T}_{m+1}$, set $m = m + 1$
  }
\end{minipage}
\end{algorithm}
