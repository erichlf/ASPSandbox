In this section we present a detailed computational study of the automated
goal-oriented adaptive algorithm presented in \autoref{sec:Algorithm} applied to
some time-dependent test problems. To this end we first we apply our algorithm
to a linear problem, \autoref{tst:Heat}, where the error representation is
orthogonal due to Galerkin orthogonality. Namely, we demonstrate the
effectiveness of \autoref{alg:Adaptivity} on the heat equation. Next we
demonstrate the effectiveness of \autoref{alg:Adaptivity} on non-linear
problems, where the error representation is no longer orthogonal. To this end,
we apply the algorithm to the benchmark problems presented in Sch\"afer et
al.\cite{Schaefer1996}, \autoref{tst:Cylinder2D}, \autoref{tst:Rectangle2D}, and
\autoref{tst:Rectangle3D}, where the equation of interest is the Icompressible
Navier-Stokes equation and our particular focus is on flow around bluff bodies.
Finally, in \autoref{tst:rhoNSE}, we apply our algorithm to the ever more
complex problem of Variable Density Incompressible Navier-Stokes equation, where
we focus on the solution to Rayleigh-Taylor instability.

For each test below we set the max iteration to $20$ and set the $TOL=0$. In
this way we will treat the solution obtained from the final adaptive mesh as the
so-called 'true solution' and use this solution to determine the effectivity of
the adaptive algorithm.

\subsection{Orthogonal Error Representation}
In this subsection we demonstrate the effectiveness of \autoref{alg:Adaptivity}
for problems where the global error representation is orthogonal. In these types
of problems the traditional thought is that the error representation doesn't
provide information about the error. However, in \autoref{tst:Heat}, we show
that the local error representation does, indeed, contain significant
information about the error and thus is an effective way of controlling the
error.

\begin{test}[Anisotropic Heat Equation] \label{tst:Heat}
  The heat equation is a purely parabolic linear partial differential equation
  and therefore the global error representation is zero due to Galerkin
  Orthogonality.  Thus it is an excellent choice to demonstrate the
  effectiveness of the algorithm described in \autoref{sec:Algorithm}.

  In this test problem we extend the test given by Li et al.
  \cite[Example 5.1]{Li2010}, i.e. the anisotropic diffusion equation, to the
  time dependent case. Thus we are concerned with the numerical solution of the
  anisotropic heat equation on a domain with a hole given by
  \begin{equation} \label{eq:Heat}
    \begin{split}
      u_t - \nabla \cdot (\kappa \nabla u) &= f, \quad \text{in } \Omega \\
      u &= g, \quad \text{on } \partial \Omega \\
      u(0;x,y) &= u_0(x,y)
    \end{split}
  \end{equation}
  where $\Omega = \Omega_1\setminus~\Omega_2$ is the physical domain, $f$ is the
  forcing function, $g$ is the function describing the boundary condition, and
  $\kappa$ is the diffusion tensor. For this test we take $\Omega_1 = [0,
  1]^2$, $\Omega_2 = [\frac{4}{9}, \frac{5}{9}]^2$, $f = 0$
  \begin{equation}
    g = \begin{cases}
        0 & \text{if } (x,y) \in \partial \Omega_1 \\
        0.5 - 0.5\, \cos(10\, t) & \text{if } (x,y) \in \partial \Omega_2,
    \end{cases}
    \label{eqn:BCFunction}
  \end{equation}
  and the diffusion tensor to be
  \begin{equation}
    \kappa = \begin{bmatrix} 500.5 & 499.5 \\ 499.5 & 500.5 \end{bmatrix}.
    \label{eqn:DiffusionTensor}
  \end{equation}
  The above diffusion tensor results in a diffusion process which is
  preferential along the line $x = y$.
  \textcolor{red}{Need info on discretization.}
\end{test}

\subsection{Non-Orthogonal Error Representation}
In this subsection we demonstrate the effectiveness of \autoref{alg:Adaptivity}
applied to problems where the global error representation is no longer
orthogonal. In particular, we look at applying \autoref{alg:Adaptivity} to the
Incompressible Navier-Stokes and the Variable Density Incompressible Navier
Stokes. In both cases, we solve each test case using a
Galerkin\slash~Least-Squares (GLS) stabilized cG(1)cG(1) finite element, where
the first cG(1) indicates that the test and trial functions are both piecewise
linear, while the second cG(1) indicates that the test piecewise constants and
the trial functions are continuous piecewise linear.

\subsubsection{Incompressible Navier-Stokes}
For test problems \ref{tst:Cylinder2D}, \ref{tst:Rectangle2D}, and
\ref{tst:Rectangle3D} we are concerned with the Incompressible Navier-Stokes
with constant kinematic viscosity, $\nu>0$, in a domain $\Omega\subset \R^d$,
with boundary $\partial \Omega$, i.e.
where the strong residual is given by
\begin{equation}
    \begin{split}
      \mathbf{u}_t + \left( \mathbf{u} \cdot \nabla \right) \mathbf{u} - \nu\,
          \Delta \mathbf{u} + \nabla p = \mathbf{f}, \quad \mathbf{x} \in \Omega \\
          \nabla \cdot \mathbf{u} = 0, \quad \mathbf{x} \in \Omega
    \end{split}
  \label{eqn:NSE}
\end{equation}
Given a step size $k$ and applying the cG(1)cG(1) finite element discretization
to \eqref{eqn:NSE} with $V = (v, q) \in X \subset [H^1_0(\Omega)]^d \times
H^1(\Omega)$ gives the following weak residual
\begin{equation}
  \begin{split}
    r(\bar{U}^n; V) &= \left(\mathbf{u}^n - \mathbf{u}^{n-1}\right)\,k^{-1}
        + (\left( \bar{\mathbf{u}}^n \cdot \nabla \right) \bar{\mathbf{u}}^n, v) \\
        &\quad+ \nu\, (\nabla \bar{\mathbf{u}}^n, \nabla v)
        - (p^n, \nabla \cdot v) - (\mathbf{f}, v)
        + (\nabla \cdot \bar{\mathbf{u}}^n, v) = 0
  \end{split}
  \label{eqn:WeakNSE}
\end{equation}
where $\bar{U}^n = (\bar{\mathbf{u}}^n,p)$, and $\bar{\mathbf{u}}^n =
\frac{1}{2}\left(\mathbf{u}^n + \mathbf{u}^{n-1}\right)$. Given the solution
$U=(\mathbf{u},p)$, since the elements we are concerned with (cG(1)cG(1)) have
test functions which are linear in space and constant in time the strong
residual is given by
\begin{equation}
    R(\bar{U}^n,U) = \begin{cases}
      \left(\mathbf{u} \cdot \nabla \right) \bar{\mathbf{u}}^n
        + \nabla p^n - \mathbf{f} = 0 \\
      \nabla \cdot \bar{\mathbf{u}}^n = 0.
    \end{cases}
  \label{eqn:StrongNSE}
\end{equation}
\textcolor{red}{Should $\mathbf{f}$ be an average between time steps?}
Finally, the GLS-cG(1)cG(1) discretization of \eqref{eqn:NSE} is given by
\begin{equation}
  r(\bar{U}^n,V) + SD_{\delta}^n(\bar{U}^n,V) = 0, \quad \forall V=(v,q) \in X.
  \label{eqn:G2}
\end{equation}
Here $SD_{\delta}^n$ is the GLS and is given by
\begin{equation}
  SD_{\delta}^n \equiv
    \delta_1 (\left(\bar{\mathbf{u}}^n \cdot \nabla \right) \bar{\mathbf{u}}^n
        + \nabla p^n - \mathbf{f},
      \left(\bar{\mathbf{u}}^n \cdot \nabla \right) v + \nabla q)
      + \delta_2 (\nabla \cdot \bar{\mathbf{U}}^n, \nabla \cdot v),
  \label{eqn:NSEStabilization}
\end{equation}
where $\delta_1 = \kappa_1 (k^{-1} + |\mathbf{u}^{n-1}|^2\, h^{-2})^{-1/2}$, and
$\delta_2 = \kappa_2 h$, while $\kappa_1$ and $\kappa_2$ are problem independent
constants of unit size.


\begin{test}[Flow Around a Cylinder] \label{tst:Cylinder2D}
\end{test}

\begin{test}[2D Flow Around a Square] \label{tst:Rectangle2D}
\end{test}

\begin{test}[3D Flow Around a Rectangular-Cylinder] \label{tst:Rectangle3D}
\end{test}

\subsubsection{Variable Density Incompressible Navier-Stokes}
\begin{test}[Variable Density Navier-Stokes] \label{tst:rhoNSE}
\end{test}
