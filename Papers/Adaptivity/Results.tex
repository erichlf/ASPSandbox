In this section we present a detailed computational study of the automated
goal-oriented adaptive algorithm presented in \autoref{sec:Algorithm} applied to
some time-dependent test problems. To this end we first we apply our algorithm
to a linear problem where the error representation is orthogonal due to Galerkin
orthogonality. Namely, we demonstrate the effectiveness of
\autoref{alg:Adaptivity} on the heat equation. Second, we demonstrate the
effectiveness of the new algorithm on the 2D Navier-Stokes equations
discretized using the a Galerkin/least-squares (GLS) stabilized finite element
with continuous piecewise linear basis functions (cG(1)). Third, we apply our
algorithm to the Variable Density Navier-Stokes equation discetized using the
GLS-cG(1) finite element.

\subsection{Heat Equation} \label{sse:Heat}
  The heat equation is a purely parabolic linear partial differential equation
  and therefore the global error representation is zero due to Galerkin
  Orthogonality.  Thus it is an excellent choice to demonstrate the
  effectiveness of the algorithm described in \autoref{sec:Algorithm}.

  In this test problem we are concerned with the numerical solution of the heat
  equation given by
  \begin{equation}
    u_t - \nabla \cdot (\kappa \nabla u) = f, \quad \text{in } \Omega
    \label{eq:Heat}
  \end{equation}
  subject to the Dirichlet boundary condition
  \begin{equation}
    u = g, \quad \text{on } \partial \Omega
    \label{eqn:HeatBC}
  \end{equation}
  where $\Omega = \Omega_1\slash~\Omega_2 = [0, 1]^2\slash~[\frac{4}{9},
  \frac{5}{9}]^2$ is the physical domain, $f=0$ is the forcing function,
  \begin{equation}
    g = \begin{cases}
        0 & \text{if } (x,y) \in \partial \Omega_1 \\
        0.5 + 0.5 \sin(\theta\, t) & \text{if } (x,y) \in \partial \Omega_2
    \end{cases}
    \label{eqn:BCFunction}
  \end{equation}

\subsection{Navier-Stokes} \label{sse:NSE}

\subsection{Variable Density Navier-Stokes} \label{sse:rhoNSE}
