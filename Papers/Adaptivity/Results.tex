In this section we present a detailed computational study of the automated
goal-oriented adaptive algorithm presented in \autoref{sec:Algorithm} applied to
some time-dependent test problems. To this end we first we apply our algorithm
to a linear problem, \autoref{tst:Heat}, where the error representation is
orthogonal due to Galerkin orthogonality. Namely, we demonstrate the
effectiveness of \autoref{alg:Adaptivity} on the heat equation. Then in
\autoref{tst:NSE2D} and \autoref{tst:NSE3D}, we demonstrate the effectiveness of
the new algorithm on the incompressible Navier-Stokes equations in 2D and 3D,
respectively, discretized using the a Galerkin/least-squares (GLS) stabilized
finite element with continuous piecewise linear basis functions (cG(1)). In
\autoref{tst:rhoNSE}, we apply our algorithm to the Variable Density
Navier-Stokes equation discetized using the GLS-cG(1) finite element.

\begin{test}[Anisotropic Heat Equation] \label{tst:Heat}
  The heat equation is a purely parabolic linear partial differential equation
  and therefore the global error representation is zero due to Galerkin
  Orthogonality.  Thus it is an excellent choice to demonstrate the
  effectiveness of the algorithm described in \autoref{sec:Algorithm}.

  In this test problem we extend the test given by Li et al.
  \cite[Example 5.1]{Li2010}, i.e. the anisotropic diffusion equation, to the
  time dependent case. Thus we are concerned with the numerical solution of the
  anisotropic heat equation on a domain with a hole given by
  \begin{equation} \label{eq:Heat}
    \begin{split}
      u_t - \nabla \cdot (\kappa \nabla u) &= f, \quad \text{in } \Omega \\
      u &= g, \quad \text{on } \partial \Omega \\
      u(0;x,y) &= u_0(x,y)
    \end{split}
  \end{equation}
  where $\Omega = \Omega_1\setminus~\Omega_2$ is the physical domain, $f$ is the
  forcing function, $g$ is the function describing the boundary condition, and
  $\kappa$ is the diffusion tensor. For this test we take $\Omega_1 = [0,
  1]^2$, $\Omega_2 = [\frac{4}{9}, \frac{5}{9}]^2$, $f = 0$
  \begin{equation}
    g = \begin{cases}
        0 & \text{if } (x,y) \in \partial \Omega_1 \\
        0.5 - 0.5\, \cos(10\, t) & \text{if } (x,y) \in \partial \Omega_2,
    \end{cases}
    \label{eqn:BCFunction}
  \end{equation}
  and the diffusion tensor to be
  \begin{equation}
    \kappa = \begin{bmatrix} 500.5 & 499.5 \\ 499.5 & 500.5 \end{bmatrix}.
    \label{eqn:DiffusionTensor}
  \end{equation}
  The above diffusion tensor results in a diffusion process which is
  preferential along the line $x = y$.
\end{test}

\begin{test}[2D Navier-Stokes] \label{tst:NSE2D}
\end{test}

\begin{test}[3D Navier-Stokes] \label{tst:NSE3D}
\end{test}

\begin{test}[Variable Density Navier-Stokes] \label{tst:rhoNSE}
\end{test}
