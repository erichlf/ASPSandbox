Let $\Omega \times I$ be the problem domain with $\Gamma$ the boundary of the
spatial domain $\Omega$ and $I = [0, T]$ the time domain.  Then the inviscid
linear shallow water equations are given by
\begin{equation}
  \begin{split}
    \mathbf{u}_t + f\mathbf{k} \times \mathbf{u} + g \nabla \eta = \mathbf{F}, \\
    \eta_t + H \nabla\cdot \mathbf{u} = 0,
  \end{split}
  \label{eqn:SWE}
\end{equation}
where $\mathbf{u}=[u_1,u_2]^T$ is the two-dimensional velocity vector, $\eta$ is the
deviation of the fluid elevation from the mean fluid depth $H_0$, $H$ is the
fluid depth (considered constant in this study), $mathbf{F}$ is a vector valued
forcing function, and $f$ is the Coriolis parameter, $\mathbf{k}$ is a unit
vector in the vertical direction \cite{Hanert2004, LeBlond1981, Le-Roux1998}.
Less compactly we see
\begin{equation}
  f\mathbf{k} \times \mathbf{u} := f\begin{bmatrix}
    -u_2 \\
    u_1
  \end{bmatrix}.
  \label{eqn:Coriolis}
\end{equation}
For a domain with hard boundaries
we solve these equations subject to the boundary condition
\begin{equation}
  \mathbf{u}\cdot \mathbf{n} = 0 \quad \text{on } \Gamma,
  \label{eqn:BCs}
\end{equation}
where $\mathbf{n}$ is the outward normal at the boundary,
and initial condition
\begin{equation}
  \mathbf{u}(0) = \mathbf{u}_0,\, \eta(0) = \eta_0.
  \label{eqn:IC}
\end{equation}

