Given $\mathbf{u} \in V=H^1_0(\Omega)$ and $\eta \in Q=H^1(\Omega)$ and
test functions $\mathbf{v} \in V$ and $\chi \in Q$ then the The weak form of the
SWE equations \eqref{eqn:SWE} with boundary condition \eqref{eqn:BCs}, written
in weak residual form, is given by
\begin{equation}
  \begin{split}
    r_1(\mathbf{u},\eta) &= (\mathbf{u}_t, \mathbf{v}) + f(\mathbf{k} \times
        \mathbf{u}, \mathbf{v}) - g (\eta, \nabla\cdot \mathbf{v})\\
    r_2(\mathbf{u},\eta) &= (\eta_t, \chi) + H (\nabla\cdot \mathbf{u},\chi).
  \end{split}
  \label{eqn:WeakSWE}
\end{equation}
Here $(\cdot, \cdot)$ is the standard $L^2$-inner product given by $(u,v) =
\int_{\Omega}\! u\cdot v\, dx$.

The cG(1)cG(1) method is a variant of the G2 method \cite{Johnson1998}, where
the time discretization is cG(1), continuous piecewise linear trial functions
and piecewise constant test functions, instead of a discontinuous Galerkin
method. For the spatial discretization cG(1) corresponds to continuous piecewise
linear trial and test functions and equal order finite element pairs for the
velocity and elevation elements. If we take $I = (0,T]$
