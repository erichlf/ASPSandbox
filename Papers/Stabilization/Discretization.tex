It was shown in \cite{Hanert2002} and \cite{Le-Roux1998} that the equal order
$P_1-P_1$ finite element pair lead to spurious elevation modes. In Hanert et al
\cite{Hanert2002} they showed that the linear system arrizing from the equal
order $P_1-P_1$ finite element pair resulted in ``four degrees of freedom
corresponding to four possible solutions.'' Thus one solution corresponds to the
elevation field and the other three correspond to spurious elevation modes. To
eliminate the spurious elevation modes one should filter them out. In what
follows we introduce the cG(1)cG(1) finite element discretization of the SWE,
which a weighted least squares residual, so as to eliminate the spurious modes.
The weight depends on the mesh size and thus as the mesh size decreases the
ammount of filtering is decreases. This method has been show to be very
effective and efficient when applied to the Navier-Stokes equations
\cite{Hoffman2003,Hoffman2006a,Hoffman2006b,Hoffman2011,Jansson2011}.

Given $\mathbf{u} \in V=H^1_0(\Omega)$ and $\eta \in Q=H^1(\Omega)$ and
test functions $\mathbf{v} \in V$ and $\chi \in Q$ then the The weak form of the
SWE equations \eqref{eqn:SWE} with boundary condition \eqref{eqn:BCs}, is given
by
\begin{equation}
  \begin{split}
    (\mathbf{u}_t, \mathbf{v}) + Ro^{-1}(\mathbf{k} \times
    \mathbf{u}, \mathbf{v}) - Fr^{-2} \Theta (\eta, \nabla\cdot \mathbf{v}) 
        &= (\mathbf{F}_1,\mathbf{v})\\
        (\eta_t, \chi) + \Theta^{-1} H (\nabla\cdot \mathbf{u},\chi) &= (F_2,\chi).
  \end{split}
  \label{eqn:WeakSWE}
\end{equation}
Here $(\cdot, \cdot)$ is the standard $L^2$-inner product given by $(u,v) =
\int_{\Omega}\! u\cdot v\, dx$.

The cG(1)cG(1) method is a variant of the G2 method \cite{Johnson1998}, where
the time discretization is cG(1), continuous piecewise linear trial functions
and piecewise constant test functions, instead of a discontinuous Galerkin
method. For the spatial discretization cG(1) corresponds to continuous piecewise
linear trial and test functions and equal order finite element pairs for the
velocity and elevation elements. Let $0 = t_0 < t_1 < \cdots < t_N = T$ be a
sequence of discrete time steps associated with the time intervals $I_n =
(t_{n-1},t_n]$ of length $k_n = t_n - t_{n-1}$, then the $n^{th}$
space-time slab is given by $S_n = \Omega \times I_n$. Now let $V^n \subset V,
Q^n \subset Q$ and $W^n = V^n \times Q^n$ be a finite element space consisting
of piecewise linear functions on a mesh $\mathcal{T}_n = {K}$ of mesh size
$h_n(x)$.

With the proper space in place, we seek functions $(\mathbf{u}_h, \eta_h)$ which
are continuous piecewise linear in space and time. Then the cG(1)cG(1) method
for the SWE, \eqref{eqn:SWE}, with boundary conditions \eqref{eqn:BCs} reads:
For $n = 1, \dots, N$, find $(\mathbf{u}_h^n, \eta_h^n) \equiv
(\mathbf{u}_h(t_n), \eta_h(t_n))$ with $(\mathbf{u}_h, \eta_h) \in W^n$, such
that
\begin{equation}
  \begin{split}
    &k_n^{-1}(\mathbf{u}_n - \mathbf{u}_{n-1}, \mathbf{v}) + Ro^{-1}(\mathbf{k} \times
    \bar{\mathbf{u}}, \mathbf{v}) - Fr^{-2}\Theta\,(\bar{\eta}, \nabla\cdot \mathbf{v}) -
        (\mathbf{F}_1,\mathbf{v}) \\
    &\quad+ k_n^{-1}(\eta_n - \eta_{n-1}, \chi) 
      + \Theta^{-2}H (\nabla\cdot \bar{\mathbf{u}},\chi) - (F_2,\chi)\\
    &\quad+ \delta_1 ( R_1(\bar{\mathbf{u}}_h^n, \eta_h^n),
      R_1(\mathbf{v}, \chi) + \mathbf{F}_1) \\
    &\quad+ \delta_2 (R_2(\bar{\mathbf{u}}_h^n, \eta_h^n),
        R_2(\mathbf{v}, \chi) + F_2)
  \end{split}
  \quad \forall (\mathbf{v},\chi) \in W^n,
  \label{eqn:cG1cG1}
\end{equation}
where $\bar{\mathbf{u}}_h^n = \frac{1}{2}(\mathbf{u}_h^n + \mathbf{u}_h^{n-1}),\,
\bar{\eta}_h^n = \frac{1}{2}(\eta_h^n + \eta_h^{n+1})$, with the stabilizing
terms given by
\begin{align*}
  R_1(\mathbf{u},\eta) &:= Ro^{-1}\mathbf{k} \times \mathbf{u} 
    + Fr^{-2} \Theta \nabla \eta - \mathbf{F}_1, \\
  R_2(\mathbf{u},\eta) &:= \Theta^{-1} H \nabla\cdot \mathbf{u} - F_2,
\end{align*}
$\delta_1 = \frac{RoFr^2\Theta^{-1}}{2}(k_n^{-2} + |\mathbf{u}^n|^2 h_n^{-1})^{-1/2}$, and
$\delta_2 = \frac{\Theta}{2}(k_n^{-2} + |\eta^n|^2 h_n^{-1})^{-1/2}$. Note that in
the strong residuals above ($R_1$ and $R_2$) the time derivative terms are zero
due to the choice of test functions which are piecewise constants in time.
