Given $\mathbf{u} \in V=H^1_0(\Omega)$ and $\eta \in Q=H^1(\Omega)$ and
test functions $\mathbf{v} \in V$ and $\chi \in Q$ then the The weak form of the
SWE equations \eqref{eqn:SWE} with boundary condition \eqref{eqn:BCs}, is given
by
\begin{equation}
  \begin{split}
    (\mathbf{u}_t, \mathbf{v}) + f(\mathbf{k} \times
        \mathbf{u}, \mathbf{v}) - g (\eta, \nabla\cdot \mathbf{v}) &= 0\\
    (\eta_t, \chi) + H (\nabla\cdot \mathbf{u},\chi) &= 0.
  \end{split}
  \label{eqn:WeakSWE}
\end{equation}
Here $(\cdot, \cdot)$ is the standard $L^2$-inner product given by $(u,v) =
\int_{\Omega}\! u\cdot v\, dx$.

The cG(1)cG(1) method is a variant of the G2 method \cite{Johnson1998}, where
the time discretization is cG(1), continuous piecewise linear trial functions
and piecewise constant test functions, instead of a discontinuous Galerkin
method. For the spatial discretization cG(1) corresponds to continuous piecewise
linear trial and test functions and equal order finite element pairs for the
velocity and elevation elements. Let $0 = t_0 < t_1 < \cdots < t_N = T$ be a
sequence of discrete time steps associated with the time intervals $I_n =
(t_{n-1},t_n]$ of length $k_n = t_n - t_{n-1}$, then the $n^{th}$
space-time slab is given by $S_n = \Omega \times I_n$. Now let $V^n \subset V,
Q^n \subset Q$ and $W^n = V^n \times Q^n$ be a finite element space consisting
of piecewise linear functions on a mesh $\mathcal{T}_n = {K}$ of mesh size
$h_n(x)$.

With the proper space in place, we seek functions $(\mathbf{u}_h, \eta_h)$ which
are continuous piecewise linear in space and time. Then the cG(1)cG(1) method
for the SWE, \eqref{eqn:SWE}, with boundary conditions \eqref{eqn:BCs} reads:
For $n = 1, \dots, N$, find $(\mathbf{u}_h^n, \eta_h^n) \equiv
(\mathbf{u}_h(t_n), \eta_h(t_n))$ with $(\mathbf{u}_h, \eta_h) \in W^n$, such
that
\begin{equation}
  \begin{split}
    &k_n^{-1}(\mathbf{u}_n - \mathbf{u}_{n-1}, \mathbf{v}) + f(\mathbf{k} \times
        \bar{\mathbf{u}}, \mathbf{v}) - g (\bar{\eta}, \nabla\cdot \mathbf{v})
        + k_n^{-1}(\eta_n - \eta_{n-1}, \chi) + H (\nabla\cdot \bar{\mathbf{u}},\chi) \\
    &\qquad+ \delta_1 ( R_1(\bar{\mathbf{u}}_h^n, \eta_h^n),
        R_1(\mathbf{v}, \chi))
    + \delta_2 (R_2(\bar{\mathbf{u}}_h^n, \eta_h^n),
        R_2(\mathbf{v}, \chi))
    \quad \forall (\mathbf{v},\chi) \in W^n,
  \end{split}
  \label{eqn:cG1cG1}
\end{equation}
where $\bar{\mathbf{u}}_h^n = \frac{1}{2}(\mathbf{u}_h^n + \mathbf{u}_h^{n-1}),\,
\bar{\eta}_h^n = \frac{1}{2}(\eta_h^n + \eta_h^{n+1})$, with the stabilizing
terms are giving by
\begin{align*}
  R_1(\mathbf{u},\eta) &:= f\mathbf{k} \times \mathbf{u} + g \nabla \eta, \\
  R_2(\mathbf{u},\eta) &:= H \nabla\cdot \mathbf{u},
\end{align*}
$\delta_1 = \frac{1}{2g}(k_n^{-2} + |\mathbf{u}^n|^2 h_n^{-2})^{-1/2}$, and
$\delta_2 = \frac{1}{2H}(k_n^{-2} + |\eta^n|^2 h_n^{-2})^{-1/2}$. Note that in
the strong residuals above ($R_1$ and $R_2$) the time derivative terms are zero
due to the choice of test functions which are piecewise constants in time.
