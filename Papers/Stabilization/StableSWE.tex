\documentclass{elsarticle}
\usepackage[a4paper]{geometry}
\usepackage{elf}
\usepackage{multirow}

\begin{document}
  \begin{frontmatter}
    \author[1]{Erich L Foster\corref{cor1}}
    \ead{efoster@bcamath.org}
    \ead[url]{http://www.bcamath.org/en/people/efoster}
    \cortext[cor1]{corresponding author}

    \author[1]{Johan Jansson}
    \ead{jjansson@bcamath.org}
    \ead[url]{http://www.bcamath.org/en/people/jjansson}

    \address[1]{Basque Center for Applied Mathematics, Alameda Mazarredo, 14,
      48009 Bilbao, Basque Country -- Spain}

    \title{A Stable Equal Order Finite Element Pair for the Shallow Water
    Equations of the Ocean}

    \begin{abstract}
      Equal order finite element pairs are known to have stability issues for
      the Shallow Water Equations, resulting in spurious computational modes.
      Thus, to avoid this, many researchers tend to use Taylor-Hood finite
      element pairs or lesser known element pairs such as the non-conforming
      $P_1^{NC}-P_1$ finite element. In this paper we introduce an equal order
      finite element pair based on the General Galerkin method. This method
      relies on a weighted least squares stabilization and has been shown to
      work very well in the context of Navier-Stokes and Euler equations. One of
      the defining features of this method is the weighted least squares
      stabilization, where the weight is dependent upon the mesh size. Thus, as
      the mesh size decreases the stabilization is decreased.
    \end{abstract}
  \end{frontmatter}

  \section{Introduction} \label{sec:Intro}
  The oceans have two important roles in the climate system:
\begin{inparaenum}[1)]\item They store and release heat seasonally, and \item
they transport heat around the globe by their large scale currents
\end{inparaenum} These currents play a significant roll in climate dynamics
by, in the case of sub-tropic gyres, transporting the warm waters of the
tropics poleward and in the case of sub-polar gyres transporting the cold
waters of the pole southward where they mix with the warm waters from the
tropics.

The first attempts at modelling circulation of the ocean were made after Ekman
(1905) developed his theory of Ekman pumping \cite{Ekman1905}. Later Sverdrup
(1947) developed a simple theory that relates the wind stress curl to ocean
currents \cite{Sverdrup1947}, followed by the development of simple analytical
models of wind induced ocean currents by Stommel (1948), Munk (1950), and
Charney (1948) \cite{Stommel1948,Munk1950,Charney1948}. Then in 1963 the first
numerical model of ocean circulation was developed by Bryan et al.
\cite{Bryan1963}. Since the work developed by Bryan there have been many
numerical studies of ocean circulation, however the majority of which have
relied on the Finite Difference Method (FDM). For a more detailed review see
Griffies et al. \cite{Griffies2000}. However, some \emph{Finite Element} ocean
models do exist, see for example, \cite{Fix} and \cite{Myers}. In addition to
the research stated above one can find: the QUODDY model by Ip et. al.
\cite{Ip1995}, a 3D shallow water equations model based on linear triangular
elements; the DGCOM model by Giraldo et al. \cite{GiraldoWWW}, a 2D shallow
water equation model based on the discontinuous Galerkin approach.

The Finite Element Method (FEM) offers some distinct advantages over the FDM:
the FEM allows for the easy representation of coastlines; and the FDM suffers
from an inherent rigidness of structured grids making refinement in specific
regions difficult. On the other hand the FEM allows for the use of
unstructured grids, which allows for grid refinement in areas for which
dynamics of the system have been deemed important, such as narrow straits,
islands, areas containing western boundary currents, or meso-scale eddies.
Meso-scale eddies are present in only small portions of the oceans, however
they contain significant portions of the oceans' kinetic energy. For this
reason it is important to resolve meso-scale eddies, however to resolve these
meso-scale eddies a mesh size on the order of 10km is needed. For a structured
grid, requiring a rigid mesh size, this is not feasible on any of todays
computers. Thus, being able to refine the mesh in particular areas allows for
meso-scale eddies to be resolved. Other methods which allow for unstructured
grids include techniques such as the finite volumes, finite elements, and
spectral methods. 

On the other hand, while the FEM can be used on unstructured grids, and
therefore allows for the easy refinement of meshes, for resolving such
features as meso-scale eddies and western boundary currents which move about
continuously a dynamic approach to mesh refinement is necessary.  However, it
is not known a priori where such eddies may occur and therefore an \emph{a
posteriori adaptive mesh refinement} scheme such as the method developed by
Bab{\v{s}}ka et al. in \cite{Babuska1978} can be used to refine a mesh in the
areas for which it is necessary. In particular we intend to adapt the method
developed by Hoffman et al. in \cite{Hoffman2004} to ocean models. 

The scheme developed in \cite{Hoffman2004} is referred to, by the authors, as
an \emph{Adaptive DNS/LES} where certain features of the flow are resolved in a
\emph{Direct Numerical Simulation} (DNS), while other features are left
unresolved and modelled by \emph{Large Eddy Simulation} (LES). In the case of
the LES model, the residual based stabilization is used as the sub-grid model.
This method has been shown to be very efficient at high Reynolds numbers
($Re>10^6$) \cite{Jansson2011}. This allows one to only apply an LES model
where the contribution of error in a desired quantity is small and therefore
not significant to the desired quantity.

However, the effects of rotation are expected to introduce new challenges and
therefore it is unlikely one can simply apply this method without some
adaptations. The majority of problems this method has been applied have been
for aerodynamics and thus many of the dynamic features of concern are not
continuously moving. Therefore, the authors haven't been concerned with mesh
coarsening. The added feature of mesh coarsening will be essential for the
efficient simulations of the world's oceans.


  \section{Shallow Water Equations} \label{sec:SWE}
  \documentclass{elsarticle}
\usepackage{elf}

\author{Erich L Foster}

\begin{document}
  \begin{frontmatter}
    \author[1]{Erich L Foster\corref{cor1}}
    \ead{efoster@bcamath.org}
    \ead[url]{http://www.bcamath.org/en/people/efoster}
    \cortext[cor1]{corresponding author}

    \author[1]{Johan Jansson}
    \ead{iliescu@vt.edu}
    \ead[url]{http://www.bcamath.org/en/people/jjansson}

    \address[1]{Basque Center for Applied Mathematics, Alameda Mazarredo, 14,
      48009 Bilbao, Basque Country -- Spain}

    \title{An adaptive DNS/LES for a Shallow Water Equation model of the ocean}

    \begin{abstract}
    \end{abstract}
  \end{frontmatter}

  \section{Introduction} \label{sec:Intro}
  The oceans have two important roles in the climate system:
\begin{inparaenum}[1)]\item They store and release heat seasonally, and \item
they transport heat around the globe by their large scale currents
\end{inparaenum} These currents play a significant roll in climate dynamics
by, in the case of sub-tropic gyres, transporting the warm waters of the
tropics poleward and in the case of sub-polar gyres transporting the cold
waters of the pole southward where they mix with the warm waters from the
tropics.

The first attempts at modelling circulation of the ocean were made after Ekman
(1905) developed his theory of Ekman pumping \cite{Ekman1905}. Later Sverdrup
(1947) developed a simple theory that relates the wind stress curl to ocean
currents \cite{Sverdrup1947}, followed by the development of simple analytical
models of wind induced ocean currents by Stommel (1948), Munk (1950), and
Charney (1948) \cite{Stommel1948,Munk1950,Charney1948}. Then in 1963 the first
numerical model of ocean circulation was developed by Bryan et al.
\cite{Bryan1963}. Since the work developed by Bryan there have been many
numerical studies of ocean circulation, however the majority of which have
relied on the Finite Difference Method (FDM). For a more detailed review see
Griffies et al. \cite{Griffies2000}. However, some \emph{Finite Element} ocean
models do exist, see for example, \cite{Fix} and \cite{Myers}. In addition to
the research stated above one can find: the QUODDY model by Ip et. al.
\cite{Ip1995}, a 3D shallow water equations model based on linear triangular
elements; the DGCOM model by Giraldo et al. \cite{GiraldoWWW}, a 2D shallow
water equation model based on the discontinuous Galerkin approach.

The Finite Element Method (FEM) offers some distinct advantages over the FDM:
the FEM allows for the easy representation of coastlines; and the FDM suffers
from an inherent rigidness of structured grids making refinement in specific
regions difficult. On the other hand the FEM allows for the use of
unstructured grids, which allows for grid refinement in areas for which
dynamics of the system have been deemed important, such as narrow straits,
islands, areas containing western boundary currents, or meso-scale eddies.
Meso-scale eddies are present in only small portions of the oceans, however
they contain significant portions of the oceans' kinetic energy. For this
reason it is important to resolve meso-scale eddies, however to resolve these
meso-scale eddies a mesh size on the order of 10km is needed. For a structured
grid, requiring a rigid mesh size, this is not feasible on any of todays
computers. Thus, being able to refine the mesh in particular areas allows for
meso-scale eddies to be resolved. Other methods which allow for unstructured
grids include techniques such as the finite volumes, finite elements, and
spectral methods. 

On the other hand, while the FEM can be used on unstructured grids, and
therefore allows for the easy refinement of meshes, for resolving such
features as meso-scale eddies and western boundary currents which move about
continuously a dynamic approach to mesh refinement is necessary.  However, it
is not known a priori where such eddies may occur and therefore an \emph{a
posteriori adaptive mesh refinement} scheme such as the method developed by
Bab{\v{s}}ka et al. in \cite{Babuska1978} can be used to refine a mesh in the
areas for which it is necessary. In particular we intend to adapt the method
developed by Hoffman et al. in \cite{Hoffman2004} to ocean models. 

The scheme developed in \cite{Hoffman2004} is referred to, by the authors, as
an \emph{Adaptive DNS/LES} where certain features of the flow are resolved in a
\emph{Direct Numerical Simulation} (DNS), while other features are left
unresolved and modelled by \emph{Large Eddy Simulation} (LES). In the case of
the LES model, the residual based stabilization is used as the sub-grid model.
This method has been shown to be very efficient at high Reynolds numbers
($Re>10^6$) \cite{Jansson2011}. This allows one to only apply an LES model
where the contribution of error in a desired quantity is small and therefore
not significant to the desired quantity.

However, the effects of rotation are expected to introduce new challenges and
therefore it is unlikely one can simply apply this method without some
adaptations. The majority of problems this method has been applied have been
for aerodynamics and thus many of the dynamic features of concern are not
continuously moving. Therefore, the authors haven't been concerned with mesh
coarsening. The added feature of mesh coarsening will be essential for the
efficient simulations of the world's oceans.


  \bibliographystyle{plain}
  \bibliography{Complete}

\end{document}


  \section{Discretization: cG(1)cG(1)} \label{sec:Discrete}
  It was shown in \cite{Hanert2002} and \cite{Le-Roux1998} that the equal order
$P_1P_1$ finite element pair lead to spurious elevation modes. In Hanert et al
\cite{Hanert2002} they showed that the linear system arrizing from the equal
order $P_1P_1$ finite element pair resulted in ``four degrees of freedom
corresponding to four possible solutions.'' Thus one solution corresponds to the
elevation field and the other three correspond to spurious elevation modes. To
eliminate the spurious elevation modes one should filter them out. In what
follows we introduce the cG(1)cG(1) finite element discretization of the SWE,
which includes stabilizations terms, so as to eliminate the spurious modes.

Given $\mathbf{u} \in V=H^1_0(\Omega)$ and $\eta \in Q=H^1(\Omega)$ and
test functions $\mathbf{v} \in V$ and $\chi \in Q$ then the The weak form of the
SWE equations \eqref{eqn:SWE} with boundary condition \eqref{eqn:BCs}, is given
by
\begin{equation}
  \begin{split}
    (\mathbf{u}_t, \mathbf{v}) + f(\mathbf{k} \times
        \mathbf{u}, \mathbf{v}) - g (\eta, \nabla\cdot \mathbf{v}) &= 0\\
    (\eta_t, \chi) + H (\nabla\cdot \mathbf{u},\chi) &= 0.
  \end{split}
  \label{eqn:WeakSWE}
\end{equation}
Here $(\cdot, \cdot)$ is the standard $L^2$-inner product given by $(u,v) =
\int_{\Omega}\! u\cdot v\, dx$.

The cG(1)cG(1) method is a variant of the G2 method \cite{Johnson1998}, where
the time discretization is cG(1), continuous piecewise linear trial functions
and piecewise constant test functions, instead of a discontinuous Galerkin
method. For the spatial discretization cG(1) corresponds to continuous piecewise
linear trial and test functions and equal order finite element pairs for the
velocity and elevation elements. Let $0 = t_0 < t_1 < \cdots < t_N = T$ be a
sequence of discrete time steps associated with the time intervals $I_n =
(t_{n-1},t_n]$ of length $k_n = t_n - t_{n-1}$, then the $n^{th}$
space-time slab is given by $S_n = \Omega \times I_n$. Now let $V^n \subset V,
Q^n \subset Q$ and $W^n = V^n \times Q^n$ be a finite element space consisting
of piecewise linear functions on a mesh $\mathcal{T}_n = {K}$ of mesh size
$h_n(x)$.

With the proper space in place, we seek functions $(\mathbf{u}_h, \eta_h)$ which
are continuous piecewise linear in space and time. Then the cG(1)cG(1) method
for the SWE, \eqref{eqn:SWE}, with boundary conditions \eqref{eqn:BCs} reads:
For $n = 1, \dots, N$, find $(\mathbf{u}_h^n, \eta_h^n) \equiv
(\mathbf{u}_h(t_n), \eta_h(t_n))$ with $(\mathbf{u}_h, \eta_h) \in W^n$, such
that
\begin{equation}
  \begin{split}
    &k_n^{-1}(\mathbf{u}_n - \mathbf{u}_{n-1}, \mathbf{v}) + f(\mathbf{k} \times
        \bar{\mathbf{u}}, \mathbf{v}) - g (\bar{\eta}, \nabla\cdot \mathbf{v})
        + k_n^{-1}(\eta_n - \eta_{n-1}, \chi) + H (\nabla\cdot \bar{\mathbf{u}},\chi) \\
    &\qquad+ \delta_1 ( R_1(\bar{\mathbf{u}}_h^n, \eta_h^n),
        R_1(\mathbf{v}, \chi))
    + \delta_2 (R_2(\bar{\mathbf{u}}_h^n, \eta_h^n),
        R_2(\mathbf{v}, \chi))
    \quad \forall (\mathbf{v},\chi) \in W^n,
  \end{split}
  \label{eqn:cG1cG1}
\end{equation}
where $\bar{\mathbf{u}}_h^n = \frac{1}{2}(\mathbf{u}_h^n + \mathbf{u}_h^{n-1}),\,
\bar{\eta}_h^n = \frac{1}{2}(\eta_h^n + \eta_h^{n+1})$, with the stabilizing
terms are giving by
\begin{align*}
  R_1(\mathbf{u},\eta) &:= f\mathbf{k} \times \mathbf{u} + g \nabla \eta, \\
  R_2(\mathbf{u},\eta) &:= H \nabla\cdot \mathbf{u},
\end{align*}
$\delta_1 = \frac{1}{2g}(k_n^{-2} + |\mathbf{u}^n|^2 h_n^{-2})^{-1/2}$, and
$\delta_2 = \frac{1}{2H}(k_n^{-2} + |\eta^n|^2 h_n^{-2})^{-1/2}$. Note that in
the strong residuals above ($R_1$ and $R_2$) the time derivative terms are zero
due to the choice of test functions which are piecewise constants in time.


  \section{Numerical Tests} \label{sec:Tests}
  \subsection{Test 1: High Frequency Elevation Modes} \label{sse:ElevModes}
  In this subsection we perform the same numerical experiment as in Hanert et al
  \cite{Hanert2002}, and Batteen and Han \cite{Batteen1981} in order to
  demonstrate the stability added by using the cG(1)cG(1) method presented in
  \autoref{sec:Discrete}. In both \cite{Hanert2002} and \cite{Batteen1981} they
  showed that certain FE element pairs, in the case of Hanert et al, or certain FD
  grids, in the case of Batteen and Han, resulted in spurious elevation modes by
  strongly forcing short wave modes.

  Following the numerical test in \cite{Hanert2002} and \cite{Batteen1981}, we
  solve the inviscid SWE equation, \eqref{eqn:SWE}, where the problem domain is
  a square with side length 1000 km with regular grid spacing of $h=50$ km,
  initial condition $\mathbf{u}_0 = 0$, and point mass source and sink of 1 m
  and -1 m at fixed locations in the middle of the domain, vertically, 500 km
  apart, horizontally, so as to force the flow.  The resulting solution are
  presented after 100 time steps. The numerical simulations were performed for
  both inertial ($R/h = 1/4$) and gravity wave ($R/h = 2$) limits, where $R$ is
  the Rossby radius of deformation given by $R \equiv \sqrt{gH}/f$. Taking the
  the Coriolis parameter to be $f = 10^{-4}$ s$^{-1}$ then we have $H \approx
  0.16$ m for the inertial limit and $H \approx 10$ m for the gravity wave
  limit. Choosing $k = 100$ s for the inertial limit and $k = 800$ s for the
  gravity wave limit, results in the same Gravitational Courant number, $C_g =
  \sqrt{gH}\,k/h$, \cite{Le-Roux1998}. A summary of all problem parameters can
  be seen in \autoref{tab:Test1Params}.

  \def\arraystretch{1.25} %add some more padding to the table
  \begin{table}
    \begin{center}
      \begin{tabular}{|c|c|c|}
        \hline
        & $R/h = 1/4$ & $R/h = 2$ \\[0.1em] \hline
        $\Omega$ & \multicolumn{2}{c|}{[1000 km, 1000 km]} \\ \hline
        $t$ & [0 s, 10\,000 s] & [0 s, 80\,000 s] \\ \hline
        $g$ & \multicolumn{2}{c|}{$9.8$ m/s$^2$} \\ \hline 
        $f$ & \multicolumn{2}{c|}{$10^{-4}$ s$^{-1}$} \\ \hline
        $H$ & 0.16 m & 10 m \\ \hline
        $k$ & 100 s & 800 s \\ \hline
        $h$ & \multicolumn{2}{c|}{50 km} \\ \hline
      \end{tabular}
      \caption{Summary of Test 1 parameters for both the gravity wave, $R/h =
      2$, and inertial wave $R/h = 1/4$, limits.}
      \label{tab:Test1Params}
    \end{center}
  \end{table}

  As was noted in Hanert et al \cite{Hanert2002} the $P_1P_1$ element, with no
  stabilization, results in a noisy elevation pattern when applied to this
  problem. %This can clearly be seen in \autoref{fig:P1P1HighFrequencyElevation}.
%  However, using the cG(1)cG(1) method results in a solution which is not noisy,
%  as can be seen in \autoref{fig:cG1cG1HighFrequencyElevation}.


  \bibliographystyle{amsalpha}
  \bibliography{Complete}

\end{document}
