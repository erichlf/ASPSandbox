\documentclass{elsarticle}
\usepackage[a4paper]{geometry}
\usepackage{elf}

\begin{document}
  \begin{frontmatter}
    \author[1]{Erich L Foster\corref{cor1}}
    \ead{efoster@bcamath.org}
    \ead[url]{http://www.bcamath.org/en/people/efoster}
    \cortext[cor1]{corresponding author}

    \author[1]{Johan Jansson}
    \ead{jjansson@bcamath.org}
    \ead[url]{http://www.bcamath.org/en/people/jjansson}

    \address[1]{Basque Center for Applied Mathematics, Alameda Mazarredo, 14,
      48009 Bilbao, Basque Country -- Spain}

    \title{A Stable Equal Order Finite Element Pair for the Shallow Water
    Equations of the Ocean}

    \begin{abstract}
      Equal order finite element pairs are known to have stability issues for
      the Shallow Water Equations, resulting in spurious computational modes.
      Thus, to avoid this, many researchers tend to use Taylor-Hood finite
      element pairs or lesser known element pairs such as the non-conforming
      $P_1^{NC}-P_1$ finite element. In this paper we introduce an equal order
      finite element pair based on the General Galerkin method. This method
      relies on a weighted least squares stabilization and has been shown to
      work very well in the context of Navier-Stokes and Euler equations. One of
      the defining features of this method is the weighted least squares
      stabilization, where the weight is dependent upon the mesh size. Thus, as
      the mesh size decreases the stabilization is decreased.
    \end{abstract}
  \end{frontmatter}

  \section{Introduction} \label{sec:Intro}
  Partial differential equations (PDEs) describe many models in accross all fields
of science, while the Finite element method (FEM) is one of the most powerful
methods for solving PDEs. However, the FEM can be both difficult to program and
difficult for many to fully grasp. Additionally, many of the associated tasks
with solving time dependent (and steady state) PDEs are the similar (if not the
same) for most PDEs and can be automated. Furthermore, many scientists may not
be familiar with programming. Thus, a package which can simplify the process of
implementing finite element (FE) solutions, including the reduction of
repetitive tasks and the amount of programming required, would be very useful.

The FEniCS project\cite{Alnae2011} set out to address many of these issues, by
offering various programming libraries (in both C++ and Python). However, the
FEniCS project, as of yet, does not automate the time-stepping procedure for
PDEs. Additionally, many of the required steps for implementing a FE solver in
FEniCS are quite repetitive and could be automated. The \ASP library sets out to
further automate the solutions of PDEs and hopefully will simplify the process.
Currently, the user is still required to do some basic programming, but we
intend to reduce this to either none or a very minimal amount.

The \ASP software library offers a simple Python programming interface, built
upon FEniCS, which will greatly reduce the amount of programming required by the
user and completely eliminate the necessity for creating a simple time stepper.
However, the user is required to setup the problem domain, boundary conditions,
forcing functions, and weak residual using the FEniCS/DOLFIN Python interface.

In addition to automating the solution to time dependent PDEs, \ASP offers a
novel automated goal oriented mesh adaptivity \cite{Foster2014e, Jansson2014a,
Jansson2014b}.  The goal oriented mesh adaptivity code is based on solving a
dual problem.  This dual problem is constructed and solved using DOLFIN-Adjoint
\cite{Ham2012, Ferrell2014}, an automated adjointing framework for FEniCS, and
thus needs very little user input.  Currently, the only additional task the user
must perform, for implementing the adaptivity, is defining a goal functional.
In this way the weak form of the PDE does not change at all and the user need
only to add the definition of a goal functional.


  \section{Shallow Water Equations} \label{sec:SWE}
  Let $\Omega \times I$ be the problem domain with $\Gamma$ the boundary of
$\Omega$ and $I = [0, T]$.  Then the inviscid linear shallow water equations are
given by \cite{Hanert2004, LeBlond1981, Le-Roux1998}
\begin{equation}
  \begin{split}
    \mathbf{u}_t + f\mathbf{k} \times \mathbf{u} + g \nabla \eta = 0, \\
    \eta_t + H \nabla\cdot \mathbf{u} = 0,
  \end{split}
  \label{eqn:SWE}
\end{equation}
where $\mathbf{u}$ is the two-dimensional velocity vector, $\eta$ is the
fluid elevation, $f$ is the Coriolis parameter, $\mathbf{k}$ is a unit
vector in the vertical direction, and $H$ is the constant depth of the fluid.
For a domain with hard boundaries we solve these equations subject to the
boundary condition
\begin{equation}
  \mathbf{u}\cdot \mathbf{n} = 0 \quad \text{on } \Gamma,
  \label{eqn:BCs}
\end{equation}
where $\mathbf{n}$ is the outward normal at the boundary,
and initial condition
\begin{equation}
  \mathbf{u}, \eta = \mathbf{u}_0, \eta_0 \quad \text{for } t = 0.
  \label{eqn:IC}
\end{equation}



  \section{Discretization: cG(1)cG(1)} \label{sec:Discrete}
  It was shown in \cite{Hanert2002} and \cite{Le-Roux1998} that the equal order
$P_1-P_1$ finite element pair lead to spurious elevation modes. In Hanert et al
\cite{Hanert2002} they showed that the linear system arrizing from the equal
order $P_1-P_1$ finite element pair resulted in ``four degrees of freedom
corresponding to four possible solutions.'' Thus one solution corresponds to the
elevation field and the other three correspond to spurious elevation modes. To
eliminate the spurious elevation modes one should filter them out. In what
follows we introduce the cG(1)cG(1) finite element discretization of the SWE,
which a weighted least squares residual, so as to eliminate the spurious modes.
The weight depends on the mesh size and thus as the mesh size decreases the
ammount of filtering is decreases. This method has been show to be very
effective and efficient when applied to the Navier-Stokes equations
\cite{Hoffman2003,Hoffman2006a,Hoffman2006b,Hoffman2011,Jansson2011}.

Given $\mathbf{u} \in V=H^1_0(\Omega)$ and $\eta \in Q=H^1(\Omega)$ and
test functions $\mathbf{v} \in V$ and $\chi \in Q$ then the The weak form of the
SWE equations \eqref{eqn:SWE} with boundary condition \eqref{eqn:BCs}, is given
by
\begin{equation}
  \begin{split}
    (\mathbf{u}_t, \mathbf{v}) + Ro^{-1}(\mathbf{k} \times
    \mathbf{u}, \mathbf{v}) - Fr^{-2} \Theta (\eta, \nabla\cdot \mathbf{v}) 
        &= (\mathbf{F}_1,\mathbf{v})\\
        (\eta_t, \chi) + \Theta^{-1} H (\nabla\cdot \mathbf{u},\chi) &= (F_2,\chi).
  \end{split}
  \label{eqn:WeakSWE}
\end{equation}
Here $(\cdot, \cdot)$ is the standard $L^2$-inner product given by $(u,v) =
\int_{\Omega}\! u\cdot v\, dx$.

The cG(1)cG(1) method is a variant of the G2 method \cite{Johnson1998}, where
the time discretization is cG(1), continuous piecewise linear trial functions
and piecewise constant test functions, instead of a discontinuous Galerkin
method. For the spatial discretization cG(1) corresponds to continuous piecewise
linear trial and test functions and equal order finite element pairs for the
velocity and elevation elements. Let $0 = t_0 < t_1 < \cdots < t_N = T$ be a
sequence of discrete time steps associated with the time intervals $I_n =
(t_{n-1},t_n]$ of length $k_n = t_n - t_{n-1}$, then the $n^{th}$
space-time slab is given by $S_n = \Omega \times I_n$. Now let $V^n \subset V,
Q^n \subset Q$ and $W^n = V^n \times Q^n$ be a finite element space consisting
of piecewise linear functions on a mesh $\mathcal{T}_n = {K}$ of mesh size
$h_n(x)$.

With the proper space in place, we seek functions $(\mathbf{u}_h, \eta_h)$ which
are continuous piecewise linear in space and time. Then the cG(1)cG(1) method
for the SWE, \eqref{eqn:SWE}, with boundary conditions \eqref{eqn:BCs} reads:
For $n = 1, \dots, N$, find $(\mathbf{u}_h^n, \eta_h^n) \equiv
(\mathbf{u}_h(t_n), \eta_h(t_n))$ with $(\mathbf{u}_h, \eta_h) \in W^n$, such
that
\begin{equation}
  \begin{split}
    &k_n^{-1}(\mathbf{u}_n - \mathbf{u}_{n-1}, \mathbf{v}) + Ro^{-1}(\mathbf{k} \times
    \bar{\mathbf{u}}, \mathbf{v}) - Fr^{-2}\Theta\,(\bar{\eta}, \nabla\cdot \mathbf{v}) -
        (\mathbf{F}_1,\mathbf{v}) \\
    &\quad+ k_n^{-1}(\eta_n - \eta_{n-1}, \chi) 
      + \Theta^{-2}H (\nabla\cdot \bar{\mathbf{u}},\chi) - (F_2,\chi)\\
    &\quad+ \delta_1 ( R_1(\bar{\mathbf{u}}_h^n, \eta_h^n),
      R_1(\mathbf{v}, \chi) + \mathbf{F}_1) \\
    &\quad+ \delta_2 (R_2(\bar{\mathbf{u}}_h^n, \eta_h^n),
        R_2(\mathbf{v}, \chi) + F_2)
  \end{split}
  \quad \forall (\mathbf{v},\chi) \in W^n,
  \label{eqn:cG1cG1}
\end{equation}
where $\bar{\mathbf{u}}_h^n = \frac{1}{2}(\mathbf{u}_h^n + \mathbf{u}_h^{n-1}),\,
\bar{\eta}_h^n = \frac{1}{2}(\eta_h^n + \eta_h^{n+1})$, with the stabilizing
terms given by
\begin{align*}
  R_1(\mathbf{u},\eta) &:= Ro^{-1}\mathbf{k} \times \mathbf{u} 
    + Fr^{-2} \Theta \nabla \eta - \mathbf{F}_1, \\
  R_2(\mathbf{u},\eta) &:= \Theta^{-1} H \nabla\cdot \mathbf{u} - F_2,
\end{align*}
$\delta_1 = \frac{RoFr^2\Theta^{-1}}{2}(k_n^{-2} + |\mathbf{u}^n|^2 h_n^{-2})^{-1/2}$, and
$\delta_2 = \frac{\Theta}{2}(k_n^{-2} + |\eta^n|^2 h_n^{-2})^{-1/2}$. Note that in
the strong residuals above ($R_1$ and $R_2$) the time derivative terms are zero
due to the choice of test functions which are piecewise constants in time.


  \section{Numerical Tests} \label{sec:Tests}
  In the following subsections we demonstrate the effectiveness of the the G2
method with cG(1)cG(1) finite elements when applied to the SWE. For each
numerical test we compare the cG(1)cG(1) discretization to the $P_1-P_1$
discretization with Crank-Nicholson time stepping scheme. The $P_1-P_1$ finite
element pair is known to produce spurious computational modes for the tests
given \cite{Le-Roux1998,Hanert2006}. Thus, any stable solutions in the following
similation are a result of the G2 cG(1)cG(1) discretization. In addition to the
stabilization we will observe the dispersivity of the cG(1)cG(1) discretization
in \autoref{sse:GravityWaves}. The dispersivity of a method is important when
considering long time integration, such as what would be seen in a OGCM
\cite{Le_Roux1998}.

\subsection{Test 1: High Frequency Elevation Modes} \label{sse:HFElevModes}
  In this subsection we perform the same numerical experiment as in Subsection
  6.1 of Hanert et al \cite{Hanert2002}, and Batteen and Han \cite{Batteen1981}
  in order to demonstrate the stability added by using the cG(1)cG(1) method
  presented in \autoref{sec:Discrete}. In both \cite{Hanert2002} and
  \cite{Batteen1981} they showed that certain FE element pairs, in the case of
  Hanert et al, or certain FD grids, in the case of Batteen and Han, resulted in
  spurious elevation modes by strongly forcing short wave modes. Thus, the
  following experiment demonstrates the effectiveness in eliminating spurious
  elevation modes.

  Following the numerical test in \cite{Hanert2002} and \cite{Batteen1981}, we
  solve the inviscid linear SWE equation, \eqref{eqn:SWE}, where the problem domain is
  a square with side length 1000 km with regular grid spacing of $h=50$ km,
  initial condition $\mathbf{u}_0 = 0$, and point mass source and sink of 1 m
  and -1 m at fixed locations in the middle of the domain, vertically, 500 km
  apart, horizontally, so as to force the flow.  The resulting solution are
  presented after 100 time steps. The numerical simulations were performed for
  both inertial ($R/h = 1/4$) and gravity wave ($R/h = 2$) limits, where $R$ is
  the Rossby radius of deformation given by $R \equiv \sqrt{gH}/f$. Taking the
  the Coriolis parameter to be $f = 10^{-4}$ s$^{-1}$ then we have $H \approx
  0.16$ m for the inertial limit and $H \approx 10$ m for the gravity wave
  limit. Choosing $k = 100$ s for the inertial limit and $k = 800$ s for the
  gravity wave limit, results in the same Gravitational Courant number, $C_g =
  \sqrt{gH}\,k/h$, \cite{Le-Roux1998}. A summary of all problem parameters can
  be seen in \autoref{tab:HFElevationParams}.

  \def\arraystretch{1.25} %add some more padding to the table
  \begin{table}[H]
    \begin{center}
      \begin{tabular}{|c|c|c|}
        \hline
        & $R/h = 1/4$ & $R/h = 2$ \\[0.1em] \hline
        $\Omega$ & \multicolumn{2}{c|}{$[0\text{ km}, 1000\text{ km}]^2$} \\ \hline
        $t$ & [0 s, 10\,000 s] & [0 s, 80\,000 s] \\ \hline
        $g$ & \multicolumn{2}{c|}{$9.8\text{ m/s}^2$} \\ \hline 
        $f$ & \multicolumn{2}{c|}{$10^{-4}\text{ s}^{-1}$} \\ \hline
        $H$ & 0.16 m & 10 m \\ \hline
        $\mathbf{u}_0(\mathbf{x})$ & \multicolumn{2}{c|}{0 m/s} \\ \hline
        $\eta_0(\mathbf{x})$ & \multicolumn{2}{c|}{0 m} \\ \hline
        $\mathbf{F}_1$ & \multicolumn{2}{c|}{ $\mathbf{0}\text{ m/s}^2$} \\ \hline
        $F_2$ & \multicolumn{2}{c|}{$
          \begin{cases} 
            1\text{ m} & \mathbf{x} = (250\text{ km}, 500\text{ km}) \\
            -1\text{ m} & \mathbf{x} = (750\text{ km}, 500\text{ km}) \\
            0\text{ m} & \text{otherwise}
          \end{cases}$} \\ \hline
        $k$ & 100 s & 800 s \\ \hline
        $h$ & \multicolumn{2}{c|}{50 km} \\ \hline
        $C_g$ & \multicolumn{2}{c|}{\textcolor{red}{Check}$\sim 0.5$} \\ \hline
      \end{tabular}
      \caption{Summary of Test 1 parameters for both the gravity wave, $R/h =
      2$, and inertial wave $R/h = 1/4$, limits.}
      \label{tab:HFElevationParams}
    \end{center}
  \end{table}

  As was noted in Hanert et al \cite{Hanert2002} the $P_1P_1$ element, with no
  stabilization, results in a noisy elevation pattern when applied to this
  problem. %This can clearly be seen in \autoref{fig:P1P1HighFrequencyElevation}.
%  However, using the cG(1)cG(1) method results in a solution which is not noisy,
%  as can be seen in \autoref{fig:cG1cG1HighFrequencyElevation}.

\subsection{Test 2: High Frequency Velocity Modes}
  In this subsection we intend to show that the cG(1)cG(1) discretization of the
  SWE is stable in the velocity field. In \cite{Hanert2002} they showed that
  certain FE element pairs resulted in spurious elevation modes by strongly
  forcing short wave modes.  To demonstrate the effectiveness of the cG(1)cG(1)
  discretization at eliminating spurious velocity modes we perform the same
  numerical experiment as in Subsection 6.2 of Hanert et al \cite{Hanert2002},
  and Batteen and Han \cite{Batteen1981}

  For this numerical experiment we take many of the same parameters as we did
  for Test 1, except the mass source, $F_2$, is set to zero, the depth, $H$, is
  set to 2000 m, the initial velocity condition is zero except at the center
  where the magnitude of the velocity is set to 1 m/s, and the time step is set
  to 100 s.
  \def\arraystretch{1.25} %add some more padding to the table
  \begin{table}[H]
    \begin{center}
      \begin{tabular}{|c|c|}
        \hline
        $\Omega$ & $[0\text{ km}, 1000\text{ km}]^2$ \\ \hline
        $t$ &  \\ \hline
        $g$ & $9.8\text{ m/s}^2$ \\ \hline 
        $f$ & $10^{-4}\text{ s}^{-1}$ \\ \hline
        $H$ & 2000 m \\ \hline
        $\mathbf{u}_0(\mathbf{x})$ & 
          $\left|\mathbf{u}\right| = 
            \begin{cases}
              1\text{ m/s} & \text{if }\mathbf{x}=(500\text{ km}, 500\text{ km}) \\ 
              0\text{ m/s} & \text{otherwise}
            \end{cases}$ \\ \hline
          $\eta_0(\mathbf{x})$ & 0 m \\ \hline
        $\mathbf{F}_1$ & 0 m/s$^2$\\ \hline
        $F_2$ & 0 m\\ \hline
        $k$ & 100 s \\ \hline
        $h$ & 50 km \\ \hline
        $C_g$ & \textcolor{red}{Check}$\sim 0.5$ \\ \hline
      \end{tabular}
      \caption{Summary of Test 2 parameters.}
      \label{tab:HFVelocityParams}
    \end{center}
  \end{table}

\subsection{Test 4: Gravity Wave Propagation and Dispersion} \label{sse:GravityWaves}
  In this subsection we explore the same numerical experiment ``a'' in Le Roux
  et al \cite{Le-Roux1998}, where the propagation and dispersion of gravity
  waves are examined in a circular basin, $\Omega$. In this test the inviscid
  linear SWE \eqref{eqn:SWE} is solved for an initial Gaussian height field
  \begin{equation}
    \eta(\mathbf{x},0) := \alpha e^{-\beta \sqrt{x_1^2 + x_2^2}},
    \label{eqn:Guass}
  \end{equation}
  initial velocity $\mathbf{u}_0(\mathbf{x})=0$, and forcing function
  $\mathbf{F}:=\mathbf{0}$. Additionally, the Coriolis parameter is set to zero,
  thus the initial Gaussian should maintain its shape throughout the simulation.

  \def\arraystretch{1.25} %add some more padding to the table
  \begin{table}[H]
    \begin{center}
      \begin{tabular}{|c|c|}
        \hline
        $\Omega$ & $\sqrt{x^2 + y^2} \le 1000$ km \\ \hline
        $t$ &  \\ \hline
        $g$ & $9.8$ m/s$^2$ \\ \hline 
        $f$ & 0 s$^{-1}$ \\ \hline
        $H$ & 2000 m \\ \hline
        $\mathbf{u}_0(\mathbf{x})$ & $\mathbf{0}$ m/s \\ \hline
        $\eta_0(\mathbf{x})$ & $\eta(\mathbf{x},0) := \alpha e^{-\beta
          \sqrt{x_1^2 + x_2^2}}$ \\ \hline
        $\alpha$ & 100 m \\ \hline
        $\beta$ & $6.4\times10^{-11}$ m$^{-2}$ \\ \hline
        $\mathbf{F}_1$ & $\mathbf{0}$ \\ \hline
        $F_2$ & 0 \\ \hline
        $k$ & 80 s \\ \hline
        $h$ & $\sim 20$ km \\ \hline
        $C_g$ & \\ \hline
      \end{tabular}
      \caption{Summary of Test 4 parameters.}
      \label{tab:GravityWaveParams}
    \end{center}
  \end{table}


  \bibliographystyle{amsalpha}
  \bibliography{Complete}

\end{document}
