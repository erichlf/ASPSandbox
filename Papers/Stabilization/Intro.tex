In 1969 the first Ocean General Circulation Model (OGCM) was developed by Bryan
et al \cite{Bryan1969}. Since the work developed by Bryan there have been many
numerical studies of OGCMs, however the majority of which have relied on the
Finite Difference Method (FDM). For a more detailed review see Griffies et al.
\cite{Griffies2000}. However, some \emph{Finite Element} (FE) ocean models do
exist, see for example, \cite{Fix} and \cite{Myers}. In addition to the the Fix
and Myers FE models one can find: the QUODDY model by Ip et. al.  \cite{Ip1995},
a 3D shallow water equations model based on linear triangular elements; the
DGCOM model by Giraldo et al. \cite{GiraldoWWW}, a 2D shallow water equation
model based on the discontinuous Galerkin approach.

The Finite Element Method (FEM) offers some distinct advantages over the FDM:
the FEM allows for the easy representation of coastlines; and the FDM suffers
from an inherent rigidness of structured grids making refinement in specific
regions difficult. On the other hand the FEM allows for the use of unstructured
grids, which allows for grid refinement in areas for which dynamics of the
system have been deemed important, such as narrow straits, islands, areas
containing western boundary currents, or meso-scale eddies.

However, it is well known that certain FE pairs allow for the existence of
spurious computational modes. A similar issue exists for the FDM, and thus the
choice of an appropriate Arakawa grid is usually discussed when developing a
FDM. Such grids aren't generally considered in the FE community, however, Hanert
et al \cite{Hanert2002} has shown that a similar consideration can be taken into
account. The more common approach is to use a FE which is of differing order,
such as Taylor-Hood FE pairs. For instance, it is well known that an equal order
FE pair implementation of the Navier-Stokes equations results in spurious
pressure modes, but a use of Taylor-Hood FE pairs does not. The explanation of
this comes from the cellebrated LBB condition \cite{Johnson2009}
\textcolor{red}{Cite more authors here}. For the rotating Shallow Water
Equations (SWE) we see a similar issue where a standard equal order FE results
in spurious elevation modes (see \cite{Hanert2002,Hanert2006,Le-Roux2005}) ,
while Taylor-Hood elements do not \cite{Hanert2004}. 

In this paper we present a least squares stabilization of a piecewise linear
equal order FE pair applied to the SWE. The SWE, have been widely used to access
various finite difference and finite element discretizations, which were
intended for OGCMs (see \cite{Batteen1981,Hanert2004,Le-Roux1998,Walters1984}.
The FE discretization we use in this paper is based on the G2 (or General
Galerkin) method, using the finite element cG(1)cG(1). This method uses
piecewise linears in space and piecewise constants in time, and will be
discussed in more detail in a later section. High-order finite elements, which
are computationally expensive have little benefit when modelling rapidly
propagating gravity waves \cite{Le-Roux1998}, and so it makes sense to explore
low-order finite element schemes. Additionally, Le Roux et al \cite{Le-Roux1998}
showed that for geostrophic balance, which is of central importance to OGCMs,
equal order linear finite element pairs were a good candidate for the
discretization of the coupled momentum-continuity equations. However, as stated
above equal order pairings of velocity-elevation FEs result in spurious
elevation modes, thus, the addition of stabilizing terms are used prevent these
spurious elevation modes. 

\textcolor{red}{Maybe we should look into whether this discretization is mass
conserving, since we intend to use it for long time integration.}

The outline of this paper is as follows: first we recall the rotating shallow
water equations for the ocean, we then discuss the basics of the G2 finite
element and then the stabilization of the cG(1)cG(1) discretization of the SWE.
Finally, We conclude the paper by demonstrating the code on some test cases.
