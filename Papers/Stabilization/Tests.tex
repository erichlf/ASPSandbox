\subsection{High Frequency Elevation Modes} \label{sse:ElevModes}
  In this section we perform the same numerical experiment as in Hanert et al
  \cite{Hanert2002}, and Batteen and Han \cite{Batteen1981} in order to
  demonstrate the stability added by using the cG(1)cG(1) method presented in
  \autoref{sec:Discrete}. In both \cite{Hanert2002} and \cite{Batteen1981} they
  showed that certain FE element pairs, in the case of Hanert et al, or certain FD
  grids, in the case of Batteen and Han, resulted in spurious elevation modes by
  strongly forcing short wave modes.

  Following the numerical test in \cite{Hanert2002} and \cite{Batteen1981}, we
  solve the inviscid SWE equation, \eqref{eqn:SWE}, where the problem domain is
  a square with side length 1000 km, initial condition $\mathbf{u}_0 = 0$, and
  point mass source and sink of 1 m and -1 m at fixed locations in the middle of
  the domain, vertically, 500 km apart, horizontally, so as to force the flow.
  The resulting solution are presenting after 100 time steps.
