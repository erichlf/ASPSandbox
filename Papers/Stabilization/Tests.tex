\subsection{Test 1: High Frequency Elevation Modes} \label{sse:ElevModes}
  In this subsection we perform the same numerical experiment as in Hanert et al
  \cite{Hanert2002}, and Batteen and Han \cite{Batteen1981} in order to
  demonstrate the stability added by using the cG(1)cG(1) method presented in
  \autoref{sec:Discrete}. In both \cite{Hanert2002} and \cite{Batteen1981} they
  showed that certain FE element pairs, in the case of Hanert et al, or certain FD
  grids, in the case of Batteen and Han, resulted in spurious elevation modes by
  strongly forcing short wave modes.

  Following the numerical test in \cite{Hanert2002} and \cite{Batteen1981}, we
  solve the inviscid SWE equation, \eqref{eqn:SWE}, where the problem domain is
  a square with side length 1000 km with regular grid spacing of $h=50$ km,
  initial condition $\mathbf{u}_0 = 0$, and point mass source and sink of 1 m
  and -1 m at fixed locations in the middle of the domain, vertically, 500 km
  apart, horizontally, so as to force the flow.  The resulting solution are
  presented after 100 time steps. The numerical simulations were performed for
  both inertial ($R/h = 1/4$) and gravity wave ($R/h = 2$) limits, where $R$ is
  the Rossby radius of deformation given by $R \equiv \sqrt{gH}/f$. Taking the
  the Coriolis parameter to be $f = 10^{-4}$ s$^{-1}$ then we have $H \approx
  0.16$ m for the inertial limit and $H \approx 10$ m for the gravity wave
  limit. Choosing $k = 100$ s for the inertial limit and $k = 800$ s for the
  gravity wave limit, results in the same Gravitational Courant number, $C_g =
  \sqrt{gH}\,k/h$, \cite{Le-Roux1998}. A summary of all problem parameters can
  be seen in \autoref{tab:Test1Params}.

  \def\arraystretch{1.25} %add some more padding to the table
  \begin{table}
    \begin{center}
      \begin{tabular}{|c|c|c|}
        \hline
        & $R/h = 1/4$ & $R/h = 2$ \\[0.1em] \hline
        $\Omega$ & \multicolumn{2}{c|}{[1000 km, 1000 km]} \\ \hline
        $t$ & [0 s, 10\,000 s] & [0 s, 80\,000 s] \\ \hline
        $g$ & \multicolumn{2}{c|}{$9.8$ m/s$^2$} \\ \hline 
        $f$ & \multicolumn{2}{c|}{$10^{-4}$ s$^{-1}$} \\ \hline
        $H$ & 0.16 m & 10 m \\ \hline
        $k$ & 100 s & 800 s \\ \hline
        $h$ & \multicolumn{2}{c|}{50 km} \\ \hline
      \end{tabular}
      \caption{Summary of Test 1 parameters for both the gravity wave, $R/h =
      2$, and inertial wave $R/h = 1/4$, limits.}
      \label{tab:Test1Params}
    \end{center}
  \end{table}

  As was noted in Hanert et al \cite{Hanert2002} the $P_1P_1$ element, with no
  stabilization, results in a noisy elevation pattern when applied to this
  problem. %This can clearly be seen in \autoref{fig:P1P1HighFrequencyElevation}.
%  However, using the cG(1)cG(1) method results in a solution which is not noisy,
%  as can be seen in \autoref{fig:cG1cG1HighFrequencyElevation}.
