Partial differential equations (PDEs) describe many models in accross all fields
of science, while the Finite element method (FEM) is one of the most powerful
methods for solving PDEs. However, the FEM can be both difficult to program and
difficult for many to fully grasp. Additionally, many of the associated tasks
with solving time dependent (and steady state) PDEs are the similar (if not the
same) for most PDEs and can be automated. Furthermore, many scientists may not
be familiar with programming. Thus, a package which can simplify the process of
implementing finite element (FE) solutions, including the reduction of
repetitive tasks and the amount of programming required, would be very useful.

The FEniCS project\cite{Alnae2011} set out to address many of these issues, by
offering various programming libraries (in both C++ and Python). However, the
FEniCS project, as of yet, does not automate the time-stepping procedure for
PDEs. Additionally, many of the required steps for implementing a FE solver in
FEniCS are quite repetitive and could be automated. The \ASP library sets out to
further automate the solutions of PDEs and hopefully will simplify the process.
Currently, the user is still required to do some basic programming, but we
intend to reduce this to either none or a very minimal amount.

The \ASP software library offers a simple Python programming interface, built
upon FEniCS, which will greatly reduce the amount of programming required by the
user and completely eliminate the necessity for creating a simple time stepper.
However, the user is required to setup the problem domain, boundary conditions,
forcing functions, and weak residual using the FEniCS/DOLFIN Python interface.

In addition to automating the solution to time dependent PDEs, \ASP offers a
novel automated goal oriented mesh adaptivity \cite{Foster2014e, Jansson2014a,
Jansson2014b}.  The goal oriented mesh adaptivity code is based on solving a
dual problem.  This dual problem is constructed and solved using DOLFIN-Adjoint
\cite{Ham2012, Ferrell2014}, an automated adjointing framework for FEniCS, and
thus needs very little user input.  Currently, the only additional task the user
must perform, for implementing the adaptivity, is defining a goal functional.
In this way the weak form of the PDE does not change at all and the user need
only to add the definition of a goal functional.
